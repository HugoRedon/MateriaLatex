\chapter{Ecuaciones}\label{chap:appendixEquations}



\begin{equation}\label{eq:pressure}
P = \frac{R T}{v-b} - \frac{a}{v^2 +u b v + w b^2 }
\end{equation}


\begin{itemize}\itemsep0ex
\item $P$ : presión en $[Pa]$.
\item $v$ : volumen molar en $[\frac{m^3}{kmol}]$
\item $a$ : Es una medida de la atracción entre las partículas. $[\frac{m^5}{kg \cdot s}]$
\item $b$ : volumen excluido por un kmol de partículas.$[\frac{m^3}{kmol}]$
\item $u$ y $w$ : Son los parámetros diferentes para cada ecuación de estado, ver tabla \ref{tab:cubics}
\item $R$ : Constante universal de los gases ideales en $\frac{m^3 Pa}{kmol K}$
\end{itemize}


\begin{equation}\label{eq:a}
	b_i = \Omega_b \frac{R T_{ci}}{p_{ci}} 
\end{equation}

\begin{equation}\label{eq:b}
 a_i = \Omega_a \frac{\left(R T_{ci}\right)^2}{p_{ci}} \alpha_i
\end{equation}


\begin{equation}\label{eq:z}
z= \frac{P V}{R T}
\end{equation}

\begin{equation}\label{eq:volume}
V = \frac{R T}{z P}
\end{equation}

\begin{equation}\label{eq:AB}
A=\frac{ap}{(RT)^2}
\qquad
B=\frac{bp}{RT}
\end{equation}

\begin{equation}
z^3-\left[1-(u-1)B\right]z^2+ \\ \left[A-uB-uB^2 +\\ wB^2\right]z-\left[AB+wB^2+wB^3\right]=0
\end{equation}



\section{Solución de la ecuación de estado cúbica}\label{sec:cubicsolution}



\begin{equation}
z= \frac{P V}{R T}
\qquad
A=\frac{ap}{(RT)^2}
\qquad
B=\frac{bp}{RT}
\end{equation}

\begin{equation}
z^3-\left[1-(u-1)B\right]z^2+ \\ \left[A-uB-uB^2 +\\ wB^2\right]z-\left[AB+wB^2+wB^3\right]=0
\end{equation}


\begin{align}
\alpha &= 1-(u-1)B\\
\beta &= A -uB-uB^2+wB^3\\
\gamma &= AB +wB^2+ 2B^3\\
C &= 3\beta - \alpha^2\\
D&= - \alpha^3+ 4.5 \alpha \beta -13.5 \gamma\\
Q&=C^3+D^2
\end{align}


\begin{itemize}
\item Si $Q \leq 0 \qquad \vartheta = \arccos \left[\frac{-D}{\sqrt{-C^3}}\right]$
\begin{description}
\item{Líquido} $z = \frac{1}{3}\left[\alpha + 2 \sqrt{-C} \cos\left(\frac{\vartheta}{3} + 120\degree \right)\right]$
\item{Vapor} $z = \frac{1}{3}\left[\alpha + 2 \sqrt{-C} \cos\left(\frac{\vartheta}{3}\right)\right]$
\end{description}
Nota: En caso de que el z del líquido sea menor que B, entonces hay que calcularla como si fuera vapor.
\item Si $Q > 0 \qquad z = \frac{1}{3}\left[\alpha + \left(-D + \sqrt{Q}\right)^{\frac{1}{3}}+ \left(-D - \sqrt{Q}\right)^{\frac{1}{3}} \right]$
\end{itemize}




\section{Entalpía del gas ideal}
\begin{equation}\label{eq:idealgasenthalpy}
h^{\neq} = \sum_{i=1}^{nc} x_i \left[ h_i^{ref} + \int_{Tref}^{T} Cp_i^{\neq} \mathrm{d}T \right]
\end{equation}

\section{Entalpía}
\begin{equation}\label{eq:enthalpy}
h = h^{\neq} + \left[ \frac{T(\frac{\partial a}{\partial T}) - a}{b\sqrt{u^2-4w} }\right] 
\ln\left[\frac{2v+b\left(u + \sqrt{u^2-4w}\right)}{2v+b\left(u - \sqrt{u^2-4w}\right)}\right]
+ pv - RT
\end{equation}

\begin{lstlisting}[caption={Ejemplo de implementación del cálculo de entalpía en la biblioteca Materia}]
private  double calculateEnthalpy( double volume){
    double idealGasEnthalpy = calculateIdealGasEnthalpy();
    double a = calculate_a_cubicParameter();
    double b = calculate_b_cubicParameter();
    double L = cubicEquationOfState.calculateL(volume, b);
    double partial_aPartial_temperature = partial_aPartial_temperature( );
    
    return idealGasEnthalpy + ((partial_aPartial_temperature - a)/b) * L  + pressure * volume - Constants.R *temperature;
}
\end{lstlisting}	




\section{Entropía}
\begin{equation}\label{eq:entropy}
s = s^{\neq} + R\ln\left[\frac{z(v-b)}{v}\right] + \frac{\frac{\partial a}{\partial T}}{b \sqrt{u^2 - 4w}}
\ln\left[\frac{2v+b\left(u + \sqrt{u^2-4w}\right)}{2v+b\left(u - \sqrt{u^2-4w}\right)}\right]
\end{equation}
\begin{equation}\label{eq:idealgasentropy}
s^{\neq} = \sum_{i=1}^{nc} x_i\left[s_i^{ref} + \int_{Tref}^T \frac{Cp_i^{\neq}}{T} \mathrm{d}T 
- R\ln \left(\frac{p}{p_{ref}}\right)- R\ln{x_i}
\right]
\end{equation}


\begin{equation}\label{eq:gibbs}
g = h - T * s;
\end{equation}


\begin{multline}\label{eq:fugacity}
\ln\hat{\phi_i} = - \ln\left(\frac{v-b}{v}\right) 
+ (z-1)\left[\frac{1}{b}\frac{\partial bN}{\partial N_i}\right]
+ \frac{a}{RTb\sqrt{u^2-4w}}
\\
\left[\frac{1}{b}\frac{\partial bN}{\partial N_i}
- \frac{1}{aN}\frac{\partial aN^2}{\partial N_i}\right]
\ln\left[\frac{2v+b\left(u + \sqrt{u^2-4w}\right)}{2v+b\left(u - \sqrt{u^2-4w}\right)}\right]
-\ln{z}
\end{multline}




\begin{equation}\label{eq:pressureacentricfactor}
 P\degree= P_c 10^{\displaystyle\left[\left(-\frac{7}{3}\right) \left(1+\omega \right)  \left(\left(\frac{T_c}{T}\right) - 1 \right) \right]}
\end{equation}


\section{Funciones objetivo para la estimación de parámetros}
Error relativo para el dato i
\begin{equation}\label{eq:relativeerror}
ER_i=\frac{P_i^{calculada} - P_i^{experimental}}{P_i^{experimental}}
\end{equation}

Error total
\begin{equation}\label{eq:totalerror}
	E_{total} = \sum_{i=1}^{nd} ER_i^2
\end{equation}

$nd$: numero de datos



\section{Presión de vapor}

\subsection{Ecuación 101}
\begin{equation}\label{eq:101vaporpressure}
	P =e^{\displaystyle\left(A +\frac{B}{T}+C\ln{T}+D {T^E}\right)}
\end{equation}
\subsection{Ecuación 10}
\begin{equation}\label{eq:10vaporpressure}
	P =e^{\displaystyle\left(A -\frac{B}{ C +T}\right)}
\end{equation}

\section{Capacidad calorífica}
\subsection{Ecuación 107 DIPPR}
\begin{equation}\label{eq:dippr107}
	C_p =A + B \left(\frac{ \frac{C}{T}}{ \sinh{\frac{D}{T}}}  \right)^2+ E \left(\frac{ \frac{E}{T}}{ \cosh{\frac{G}{T}}}  \right)^2
\end{equation}
\subsection{Ecuación chempsep 16}
\begin{equation}\label{eq:chempse16}
	C_p = A + e^{\displaystyle\left( \frac{B}{T} + C + D T + E T^2 \right)}
\end{equation}