\subsection{Factor de compresibilidad}\label{subsec:compresibilityFactor}

Es necesario realizar la solución de la ecuación de estado cúbica para conocer el factor de compresibilidad.

\begin{equation}
z= \frac{P V}{R T}
\qquad
A=\frac{ap}{(RT)^2}
\qquad
B=\frac{bp}{RT}
\end{equation}

\begin{equation}
z^3-\left[1-(u-1)B\right]z^2+ \\ \left[A-uB-uB^2 +\\ wB^2\right]z-\left[AB+wB^2+wB^3\right]=0
\end{equation}


La solución de la ecuación se realiza en el método ``calculateCompresibilityFactor()'' de la clase ``Cubic''.

El método recibe los parámetros adimensionales A, B y la fase a la cual se desea calcular el factor de compresibilidad.

La clase ``Cubic'' tiene los métodos necesarios para transformar los parámetros a y b a su forma adimensional A y B.

\begin{lstlisting}[label=se,caption={Cálculo del factor de compresibilidad, y adimensionamiento de los parámetros a y b con la clase ``Cubic''}]
Cubic cubic = EquationsOfState.vanDerWaals();
		
double criticalTemperature = 540.2;
double criticalPressure = 2.74000E+06;

double pressure = criticalPressure * 1.5;
double reducedTemperature= criticalTemperature * 2;

//parametros de van Der Waals para el heptano
double a = 3107000.0;
double b = 0.2049;

double A =cubic.get_A(temperature, pressure, a);
double B = cubic.get_B(temperature, pressure, b);

double z =cubic.calculateCompresibilityFactor(A, B, Phase.LIQUID);
\end{lstlisting}
 
	Variando las presión y la temperatura podemos formar los diagramas de la figura \ref{fig:zchart}.

\begin{figure}
\begin{tabular}{c c}
	\begin{tikzpicture}
	\begin{axis}[width=0.45\linewidth,font=\footnotesize,view/v=-6,
		ylabel= {Presión reducida },
		xlabel= {Temperatura reducida},
		zlabel={Factor de compresibilidad z}]%[colormap/hot]
	\addplot3[surf,point meta=explicit] table[meta=rt,x=p,y=rt,z=z]{plotdata/compresibilitiChart/pz_temp.dat};
	\end{axis}
	\end{tikzpicture}
	&
	\begin{tikzpicture}
	\begin{axis}[width=0.45\linewidth,font=\footnotesize,
		xlabel= {Presión reducida },
		ylabel= {Factor de compresibilidad z}]%[colormap/hot]
	\addplot[blue]table{plotdata/compresibilitiChart/pz_temp.dat};
	\end{axis}
	\end{tikzpicture}
\end{tabular}
\caption{Diagrama del factor de compresibilidad }
\label{fig:zchart}
\end{figure}
