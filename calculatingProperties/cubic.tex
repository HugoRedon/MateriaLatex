\section{Ecuación de estado cúbica}
 
 \subsection{Cálculo de presión}
 \label{subsec:pressure}

\begin{equation}
	P = \frac{R T}{v-b} - \frac{a}{v^2 +u b v + w b^2 }
\end{equation}

{$P$} : presión en [Pa].\\
{$v$} : volumen molar en [$\frac{m^3}{kg}$]\\
{$a$} : Es una medida de la atracción entre las partículas. [$\frac{m^5}{kg s}$]\\
{$a$} : volumen excluido por un mol de partículas.[$\frac{m^3}{kg}$]\\
{$u $ y $w $} : Son los parámetros diferentes para cada ecuación de estado.\\

\begin{table}
\begin{tabular}{|c |c | c | c | c |}
	\hline
	Ecuación de estado  & $u$ & $w$ & $\Omega_a$&$\Omega_b$\\
	\hline
	Van Der Waals  & $0$ & $0$ & $0,421875$ & $0,125$\\
	\hline
	Peng robinson  & $2$ & $-1$ & $0.45723553$ & $0.077796074$\\
	\hline
	Redlich Kwong  & $1$ & $0$ & $0.42748023$ & $0.08664035$\\
	\hline
	TST  & $2.5$ & $-1.5$ &$ 0.470507$ & $0.0740740$\\
	\hline
\end{tabular}
\caption{Ecuaciónes de estado cúbicas}\label{tab:cubics}
\end{table}

La ecuación de estado cúbica esta representada por la clase Cubic en la librería de funciones Materia.

Un cálculo de presión con la libreria, se realiza de la siguiente manera.
\begin{lstlisting}[label=pressureCalculation,caption=Cálculo de presión para el heptano con la ecuación de estado cúbica de Van Der Waals]
import termo.eos.Cubic;
...
Cubic cubic = new Cubic();
//parametros de van der waals para el heptano
double a = 3107000.0;
double b = 0.2049;

double volume = 1.5;
double temperature = 300;
double pressure = cubic.calculatePressure(temperature, volume, a, b);
\end{lstlisting}

\begin{figure}
\begin{tabular}{c c}
	\begin{tikzpicture}
	\begin{axis}[width= 0.45 \linewidth]
	\addplot[blue]table{plotdata/pressurevolume.dat};
	\end{axis}
	\end{tikzpicture}
	&

	\begin{tikzpicture}
	\begin{axis}[width= 0.45 \linewidth]
	\addplot3[surf,
	colormap={blueblack}{color=(white) color=(blue)},
	domain=0:1]table{plotdata/pressurevolumetemperature.dat};
	\end{axis}
	\end{tikzpicture}
\end{tabular}

\caption{Diagrama de presión contra volumen molar usando la eq. Van Der Waals (u=w=0)} \label{fig:cubicPressureDiagrams}
\end{figure}


De manera predeterminada los valores u y w de la ecuación de estado son iguales a 0;

Podemos elegir los valores u y w para la ecuación de estado cúbica, o podemos utilizar la clase "EquationsOfState" para obtener una ecuación predeterminada.

\begin{lstlisting}[label=pengRobinsonCreation,caption=Creación de la ecuación de estado de Peng Robinson usando los metodos Set de los parametros u y w]
Cubic pengRobinson = new Cubic();
pengRobinson.setU(2);
pengRobinson.setW(-1);
\end{lstlisting}

\begin{lstlisting}[label=tstCreation,caption=Creación de la ecuación de estado de TST usando la clase EquationsOfState]
Cubic tst = EquationsOfState.twoSimTassone();
\end{lstlisting}












