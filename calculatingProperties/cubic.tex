\section{Ecuación de estado cúbica}

La ecuación de estado cúbica representada por la clase ``Cubic'', permite realizar los cálculos de: 
\begin{itemize}
\itemsep0ex
	\item{\nameref{subsec:pressure}} 
	\item{\nameref{subsec:compresibilityFactor}}
	\item{Volumen molar}
	\item{Fugacidad}
	\item{Adimensionalización de los parámetros a y b}
\end{itemize}

 \subsection{Presión}
 \label{subsec:pressure}

Un cálculo de presión, se realiza como se muestra en el código \ref{lstpressureCalculation}.
\begin{lstlisting}[label=lst:pressureCalculation,caption=Cálculo de presión para el heptano con la ecuación de estado cúbica de Van Der Waals]
import termo.eos.Cubic;
...
Cubic cubic = new Cubic();
//parametros de van der waals para el heptano
double a = 3107000.0;
double b = 0.2049;

double volume = 1.5;
double temperature = 300;
double pressure = cubic.calculatePressure(temperature, volume, a, b);
\end{lstlisting}

\begin{figure}
\begin{tabular}{c c}
	\begin{tikzpicture}
	\begin{axis}[width= 0.45 \linewidth,font=\footnotesize,
	xlabel = {Volumen molar $[\frac{m^3}{kg}]$},
	ylabel = {Presión $[Pa]$}]
	\addplot[blue]table{plotdata/pressurevolume.dat};
	\end{axis}
	\end{tikzpicture}
	&
	\begin{tikzpicture}
	\begin{axis}[width= 0.45 \linewidth,,font=\footnotesize,
	xlabel={Volumen molar $[\frac{m^3}{kg}]$},
	zlabel={Presión $[Pa]$},
	ylabel={Temperatura $[K]$}]
	\addplot3[surf,
	colormap={blueblack}{color=(white) color=(blue)},
	domain=0:1]table{plotdata/pressurevolumetemperature.dat};
	\end{axis}
	\end{tikzpicture}
\end{tabular}
\caption{Diagramas de presión usando la eq. de Van Der Waals (u=w=0)} \label{fig:cubicPressureDiagrams}
\end{figure}


De manera predeterminada los valores u y w de la ecuación de estado son iguales a 0. En la figura \ref{fig:cubicPressureDiagrams} se muestra un ejemplo de uso del cálculo para realizár gŕaficas de presión.

Podemos asignar los valores u y w para la ecuación de estado cúbica, o podemos utilizar la clase ``EquationsOfState'' para obtener una ecuación con los parámetros previamente establecidos, los fragmentos de código \ref{lst:pengRobinsonCreation} y \ref{lst:tstCreation} muestran el procedimiento respectivamente.

\begin{lstlisting}[label=lst:pengRobinsonCreation,caption=Creación de la ecuación de estado de Peng Robinson usando los metodos `Set' de los parametros u y w]
Cubic pengRobinson = new Cubic();
pengRobinson.setU(2);
pengRobinson.setW(-1);
\end{lstlisting}

\begin{lstlisting}[label=lst:tstCreation,caption=Creación de la ecuación de estado de TST usando la clase EquationsOfState]
Cubic tst = EquationsOfState.twoSimTassone();
\end{lstlisting}












