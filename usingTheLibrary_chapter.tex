\chapter{Uso de la librería}\label{chap:libraryUse}
	
	Dada la extensión de la librería, no resulta posible ni deseable mostrar el código fuente completo\footnote{El código completo puede ser consultado y descargado desde la página \url{https://github.com/HugoRedon/Materia}. El apéndice \ref{chap:github} guía al lector en el procedimiento de modificación o extensión del código fuente.}, en este capítulo se presentan pequeños fragmentos de código que muestran la forma de utilizar la librería y su estructura.
		
	La clase `Cubic' realiza cálculos de presión, fugacidad, factor de compresibilidad y adimensionamiento de los parámetros $a$ y $b$, pero no calcula los parámetros $a$ y $b$.Para calcular los parámetos $a$ y $b$ existen las clases `Substance' y `Mixture' que definen el cálculo de los parámetros para un compuesto puro o para una mezcla respectivamente. 

	Las clases `Substance' y `Mixture' tienen en común los métodos `calculate\_a\_cubicParameter',`calculate\_b\_cubicParameter', ademas de otros que se condensan en la clase `Homogeneous'.La clase `Homogeneous' utiliza los cálculos de los parámetros de la ecuación cúbica y la clase cúbic para finalmente realizar el cálculo de la fugacidad, presión ,factor compresibilidad, entalpía, entropía y energía libre de gibbs.

	La clase `Homogeneous' no realiza cálculos de equlibrio, ya que la clase representa una sola fase. Para realizar cálculos de equilibrio existe la clase `Heterogeneous' que contiene dos fases una líquida y una vapor. A través de los cálculos de fugacidad de cada fase y empleando un algoritmo numérico se pueden realizar los cálculos de equilibrio Líquid-Vapor.


	Las secciones del capítulo:
	\begin{itemize}
		\item{Sección} \ref{sec:units} Se definen las unidades que emplearan durante todo el capítulo.
		\item{Sección} \ref{sec:cubic}  La clase `Cubic' realiza cálculos de presión, fugacidad, factor de compresibilidad y volumen molar.
		\item{Sección} \ref{sec:compounds} La clase `Compound' para definir y utilizar las propiedades del compuesto puro dentro de la librería.
		\item {Sección} \ref{sec:parameters} La clase `Homogeneous' y sus implementaciones `Substance' y `Mixture' para calcular los parámetros de la ecuación cúbica.
		\item {Sección} \ref{sec:homogeneous} La clase `Homogeneous' para realizar cálculos de entalpía, entropía y energía libre de Gibbs.
		\item {Sección} \ref{sec:heterogeneous} La clase `Heterogeneous' y sus implementaciónes `HeterogeneousSubstance' y `HeterogeneousMixutre' para realizar cálculos de equilibrio Líquido-Vapor.
	\end{itemize}
