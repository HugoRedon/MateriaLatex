\chapter{Uso de la librería}\label{chap:libraryUse}
	
	Dada la extensión de la librería, no resulta posible ni deseable mostrar el código fuente completo\footnote{El código completo puede ser consultado y descargado desde la página \url{https://github.com/HugoRedon/Materia}. El apéndice \ref{chap:github} guía al lector en el procedimiento de modificación o extensión del código fuente.} en este capítulo, se mostrarán pequeños fragmentos de código mostrando la forma de utilizar la librería y su estructura.
		
	Los temas a tratar en este capítulo:
	\begin{itemize}
		\item{Sección} \ref{sec:cubic}  La clase `Cubic' realiza cálculos de presión, fugacidad, factor de compresibilidad y adimensionamiento de los parámetros $a$ y $b$.
		\item{Sección} \ref{sec:compounds} La clase `Compound' para definir y utilizar las propiedades del compuesto puro dentro de la librería.
		\item {Sección} \ref{sec:parameters} La clase `Homogeneous' y sus implementaciones `Substance' y `Mixture' para calcular los parámetros de la ecuación cúbica.
		\item {Sección} \ref{sec:homogeneous} La clase `Homogeneous' para realizar cálculos de entalpía, entropía y energía libre de Gibbs.
		\item {Sección} \ref{sec:heterogeneous} La clase `Heterogeneous' y sus implementaciónes `HeterogeneousSubstance' y `HeterogeneousMixutre' para realizar cálculos de equilibrio Líquido-Vapor.

	\end{itemize}
