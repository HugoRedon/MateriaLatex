\chapter{Uso de la librería}\label{chap:libraryUse}
	
	Dada la extensión de la librería, no resulta posible ni deseable mostrar el código fuente completo\footnote{El código completo puede ser consultado y descargado desde la página \url{https://github.com/HugoRedon/Materia}. El apéndice \ref{chap:github} guía al lector en el procedimiento de modificación o extensión del código fuente.} en este documento, en cambio se mostrarán pequeños fragmentos de código mostrando la forma de utilizar la librería y su estructura.
		

	\begin{itemize}
		\item En la sección \ref{sec:cubic} se muestra como utilizar la clase `Cubic'para realizar cálculos de presión, fugacidad, factor de compresibilidad y adimensionamiento de los parámetros $a$ y $b$.
		\item En la sección \ref{sec:compounds} se muestra como utilizar la clase `Compound' para definir las propiedades del compuesto puro utilizarlas dentro de la librería.
		\item En la sección \ref{sec:parameters} se muestra como utilizar la clase `Homogeneous' y sus implementaciones `Substance', `Mixture' para calcular los parámetros de la ecuación cúbica.
		\item En la sección \ref{sec:homogeneous} se muestra como utilizar la clase `Homogeneous' para realizar cálculos de entalpía, entropía y energía libre de Gibbs.
	\end{itemize}
