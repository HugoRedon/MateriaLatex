\subsection{Factor de compresibilidad}\label{subsec:compresibilityFactor}

En el fragmento de código \ref{lst:compresibility} se muestra el cálculo de el factor de compresibilidad para una substancia o mezcla homogénea.

\begin{lstlisting}[label=lst:compresibility,caption={Cálculo del factor de compresibilidad, y adimensionamiento de los parámetros a y b con la clase ``Cubic''}]
substance.setPressure(pressure);
substance.setTemperature(temperature);
double z = substance.calculateCompresibilityFactor();
\end{lstlisting}
 
	Para un rango de presión y a diferentes temperaturas podemos formar los diagramas de la figura \ref{fig:zchart}.

\begin{figure}
\begin{tabular}{c c}
	\begin{tikzpicture}
	\begin{axis}[width=0.45\linewidth,font=\footnotesize,view/v=-6,
		ylabel= {Presión reducida },
		xlabel= {Temperatura reducida},
		zlabel={Factor de compresibilidad z}]%[colormap/hot]
	\addplot3[surf,point meta=explicit] table[meta=rt,x=p,y=rt,z=z]{plotdata/compresibilitiChart/pz_temp.dat};
	\end{axis}
	\end{tikzpicture}
	&
	\begin{tikzpicture}
	\begin{axis}[width=0.45\linewidth,font=\footnotesize,
		xlabel= {Presión reducida },
		ylabel= {Factor de compresibilidad z}]%[colormap/hot]
	\addplot[blue]table{plotdata/compresibilitiChart/pz_temp.dat};
	\end{axis}
	\end{tikzpicture}
\end{tabular}
\caption{Diagramas del factor de compresibilidad con la ecuación de estado de Van Der Waals para el heptano}
\label{fig:zchart}
\end{figure}
