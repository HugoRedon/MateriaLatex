\subsection{Temperatura de Burbuja}\label{subsec:bubbletemperature}

	El cálculo de temperatura de burbuja para una mezcla se realiza como se indica en la figura \ref{fig:bubbletemperature}. Usando un objeto del tipo `HeterogeneousMixture' de la librería \Materia el cálculo se realiza como se muestra en los fragmentos de código \ref{lst:bubbletemperature} y \ref{lst:bubbletemperatureWithEstimate}.

	Si al método no se le proporciona un estimado inicial, como se muestra en el código \ref{lst:bubbletemperature} entonces la clase realiza la estimación de la temperatura como se muestra en la figura \ref{fig:bubbletemperatureEstimate}. 

	\begin{lstlisting}[label={lst:bubbletemperatureWithEstimate},caption={Cálculo de la temperatura de burbuja proporcionando un estimado inicial.}]

		heterogeneousMixture.setZFraction(compound1,molarFraction1);
		//...... asignar la fracción molar para todos los compuestos

		heterogeneousMixture.setPressure(pressure);
		heterogeneousMixture.bubbleTemperature(temeperatureEstimate);
		double temperature = heterogeneousMixture.getTemperature();
	\end{lstlisting}


	\begin{lstlisting}[label={lst:bubbletemperature},caption={Cálculo de la temperatura de burbuja.}]

		heterogeneousMixture.setZFraction(compound1,molarFraction1);
		//...... asignar la fracción molar para todos los compuestos

		heterogeneousMixture.setPressure(pressure);
		heterogeneousMixture.bubbleTemperature();
		double temperature = heterogeneousMixture.getTemperature();
	\end{lstlisting}


\begin{figure}[!h]
	\centering	
	\begin{tikzpicture}
	\begin{axis}[xlabel={Temperatura [K]},ylabel={Presión [Pa]}]
	\addplot[blue]table{plotdata/heterogeneous/et.dat};
	\end{axis}
	\end{tikzpicture}
	\caption{}\label{fig:bubblePressure}
\end{figure}

\begin{tikzpicture}
\begin{axis}[view/h=-165,xlabel={Temperatura[K]},ylabel={Volumen Molar $[\frac{m³}{kmol}]$}]
\addplot3[surf,point meta=explicit]table[meta=temperature, x=temperature , y=liquidVolume, z=pressure]{plotdata/heterogeneous/et.dat};
\addplot3[surf,point meta=explicit]table[meta=temperature, x=temperature , y=vaporVolume, z=pressure]{plotdata/heterogeneous/et.dat};
\end{axis}
\end{tikzpicture}



\begin{figure}[!h]
	\begin{tikzpicture}[nodes={draw, fill=white,align=center},row sep=1.4cm,column sep=0.7cm] ]

	\node(init){Inicio};
	\node[below of=init,below] (lab)  {Lectura de \textbf{Datos} \\$P,x_1, x_2,\ldots, x_{nc}$};
	\node[below of=lab,below = 0.4cm](estim){Estimado inicial de\\ las incognitas\\$T,y_1,y_2,\ldots,y_{nc}$};
	\node[below of=estim,below = 0.6](relations){Cálculo de las\\ razones de equilibrio\\
	$K_i = \frac{ \hat{\phi}_i^L }{\hat{ \phi}_i^V}$};
	\node[below of=relations,below = 0.6cm](error){Cálculo de la función Error\\$ S_y =\sum_{i=1}^{nc} K_i x_i $\\$ E = \ln{S_y} $};

	\node[below of=error,below = 0.6](criteria){$|E| \leq 1\cdot10^{-4}\quad \text{o} \quad  1\cdot 10^{-5}$};
	\node[below of=criteria,below = 0.4](tempIncrement){Incrementer la temperatura\\$T^* = T + \Delta T$\\
	$\Delta T = 0.1  \quad \text{o} \quad 1.0 K$};

	\node[below of=tempIncrement,below = 0.4](relationsWithIncrement){Cálculo de las Razones\\ de Equilibrio con $T^*$
	\\$K_i^* = \frac{ \hat{\phi}_i^L }{ \hat{\phi}_i^V}$};

	\node[right of=criteria	,right=2.2cm](end){Fin};


	\node[right of=tempIncrement, right=3cm](errorWithIncrement){Cálculo de la función Error \\ con $T^*$\\
	$ S_y^* =\sum_{i=1}^{nc} K_i^* x_i $\\$ E^* = \ln{S_y^*} $};

	\node[right of=relations, right = 3cm](newValues){Cálculo de las nuevas estimaciones \\ de las \textbf{Incógnitas}\\
	\begin{minipage}{0.2\linewidth}
	\begin{equation*}
	T_{nueva} = \frac{TT^* \left(E^*-E\right)}{T^*E^*-TE}
	\end{equation*}
	\begin{equation*}
	 S_y =\sum_{i=1}^{nc} K_i x_i \\
	 \end{equation*}
	 \begin{equation*}
	\left(y_i\right)_{nueva} = \frac{K_i x_i}{S_y}
	\end{equation*}
	\end{minipage}
	};


	\draw[-latex] (init)--(lab);
	\draw[-latex] (lab)--(estim);
	\draw[-latex] (estim)--(relations);
	\draw[-latex] (relations)--(error);
	\draw[-latex] (error)--(criteria);
	\draw[-latex] (criteria)--node[fill=none,draw=none,above]{SI}(end);
	\draw[-latex] (criteria)--node[fill=none,draw=none,left]{NO}(tempIncrement);
	\draw[-latex] (tempIncrement)--(relationsWithIncrement);
	\draw[-latex] (relationsWithIncrement)--(errorWithIncrement);
	\draw[-latex] (errorWithIncrement)--(newValues);
	\draw[-latex] (newValues)--(relations);

	\end{tikzpicture}
	\caption{Algoritmo para el cálculo de temperatura de burbuja.}\label{fig:bubbletemperature}
\end{figure}










\begin{figure}[!h]
	\begin{tikzpicture}[nodes={draw, fill=white,align=center},row sep=1.4cm,column sep=0.7cm] ]

	\node(init){Inicio};
	\node[below of=init,below] (lab)  {Lectura de \textbf{Datos} \\$P,x_1, x_2,\ldots, x_{nc}$};
	\node[below of=lab,below = 0.4cm](estim){Estimado inicial de\\ la incognita\\$T=300 K$};
	\node[below of=estim,below = 0.6](relations){Cálculo de la presión de vapor con T \\
	$P\degree = \sum_{i=1}^{nc} x_i p\degree_i$\\${p\degree_i}= P_c 10^{\displaystyle\left[\left(-\frac{7}{3}\right) \left(1+\omega \right)  \left(\left(\frac{T_c}{T}\right) - 1 \right) \right]}$};
	\node[below of=relations,below = 0.6cm](error){Cálculo de la función Error\\$ E = \ln{\displaystyle\frac{P\degree}{P}} $};

	\node[below of=error,below = 0.6](criteria){$|E| \leq 1\cdot10^{-4}\quad \text{o} \quad  1\cdot 10^{-5}$};
	\node[below of=criteria,below = 0.4](tempIncrement){Incrementer la temperatura\\$T^* = T + \Delta T$\\
	$\Delta T = 0.1  \quad \text{o} \quad 1.0 K$};

	\node[below of=tempIncrement,below = 0.4](relationsWithIncrement){Cálculo de la presión de vapor con $T^*$ \\
	$P\degree^* = \sum_{i=1}^{nc} x_i p\degree_i^*$\\${p\degree_i}^*= P_c 10^{\displaystyle\left[\left(-\frac{7}{3}\right) \left(1+\omega \right)  \left(\left(\frac{T_c}{T^*}\right) - 1 \right) \right]}$};

	


	\node[right of=relationsWithIncrement, right=5.5cm](errorWithIncrement){Cálculo de la función Error \\ con $T^*$\\$ E^* = \ln{\displaystyle\frac{P\degree^*}{P}} $};

	\node[right of=relations, right = 5.5cm](newValues){Cálculo de las nueva estimacion \\ de la \textbf{Incógnita}\\
	\begin{minipage}{0.2\linewidth}
	\begin{equation*}
	T_{nueva} = \frac{TT^* \left(E^*-E\right)}{T^*E^*-TE}
	\end{equation*}
	\end{minipage}
	};


	\node[right of=criteria,right=2.5cm](newfractions){Cálculo de las\\ fracciones del vapor\\
	\begin{minipage}{0.2\linewidth}

	\begin{equation*}
	 y_i = \frac{p\degree_i x_i}{P} \\
	 \end{equation*}
	\end{minipage}};


	\node[right of=newfractions	,right=1.7cm](end){Fin};


	\draw[-latex] (init)--(lab);
	\draw[-latex] (lab)--(estim);
	\draw[-latex] (estim)--(relations);
	\draw[-latex] (relations)--(error);
	\draw[-latex] (error)--(criteria);
	\draw[-latex] (criteria)--node[fill=none,draw=none,above]{SI}(newfractions);
	\draw[-latex] (newfractions)--(end);
	\draw[-latex] (criteria)--node[fill=none,draw=none,left]{NO}(tempIncrement);
	\draw[-latex] (tempIncrement)--(relationsWithIncrement);
	\draw[-latex] (relationsWithIncrement)--(errorWithIncrement);
	\draw[-latex] (errorWithIncrement)--(newValues);
	\draw[-latex] (newValues)--(relations);

	\end{tikzpicture}
	\caption{Algoritmo para el cálculo de la temperatura de burbuja con la ecuación del factor acéntrico para la presión.}\label{fig:bubbletemperatureEstimate}
\end{figure}


