\subsection{Energía libre de Gibbs}
	
	El cálculo de la energía libre de Gibbs se realiza a partir de la entropía y la entalpía segun la ecuación \ref{eq:gibbs} y su cálculo se demuestra en el código \ref{lst:gibbs}.

	\begin{lstlisting}[label={lst:gibbs},caption={Cálculo de la energía libre de Gibbs con la librería \Materia}]
		double gibbs = homogeneous.calculateGibbs();
	\end{lstlisting}

	Las figuras \ref{fig:2dgibbs} y \ref{fig:gibbs3d} muestran diagramas de la energía libre de Gibbs creados con ayuda de la librería \Materia.

\begin{figure}[!h]
	\centering	
	\begin{tikzpicture}
	\begin{axis}[xlabel={\gibbs},ylabel={\pressure}]
	\addplot[blue]table{plotdata/gibbs/lv.dat};
	\end{axis}
	\end{tikzpicture}
	\caption{Diagrama de la energía libre de Gibbs para el agua. Las líneas azules representan isotermas. Nótese el cambio en las pendientes de las líneas isotermas, este cambio indica un cambio de fase.}\label{fig:2dgibbs}
\end{figure}

\begin{figure}[!h]
	\begin{tikzpicture}
	\begin{axis}[view/h=-165,xlabel={\gibbs},ylabel={\molarVolume},zlabel={\pressure},colorbar,colorbar style={ylabel=Temperatura (K),title=Código de color}]
	\addplot3[surf,point meta=explicit]table[meta=temperature]{plotdata/gibbs/lv3d.dat};
	\addplot3[surf,point meta=explicit]table[meta=temperature]{plotdata/gibbs/l3d.dat};
	\addplot3[surf,point meta=explicit]table[meta=temperature]{plotdata/gibbs/v3d.dat};
	\end{axis}
	\end{tikzpicture}
	% \begin{tikzpicture}
	% \begin{axis}[view/h=-225,xlabel={\gibbs},ylabel={\molarVolume},zlabel={\pressure}]
	% \addplot3[surf,point meta=explicit]table[meta=temperature]{plotdata/gibbs/lv3d.dat};
	% \addplot3[surf,point meta=explicit]table[meta=temperature]{plotdata/gibbs/l3d.dat};
	% \addplot3[surf,point meta=explicit]table[meta=temperature]{plotdata/gibbs/v3d.dat};
	% \end{axis}
	% \end{tikzpicture}
	% \begin{tikzpicture}
	% \begin{axis}[view/h=-120,xlabel={\gibbs},ylabel={\molarVolume},zlabel={\pressure}]
	% \addplot3[surf,point meta=explicit]table[meta=temperature]{plotdata/gibbs/lv3d.dat};
	% \addplot3[surf,point meta=explicit]table[meta=temperature]{plotdata/gibbs/l3d.dat};
	% \addplot3[surf,point meta=explicit]table[meta=temperature]{plotdata/gibbs/v3d.dat};
	% \end{axis}
	% \end{tikzpicture}
	\caption{Diagrama tridimensional de presión-`energía libre de Gibbs'-`volumen molar' para el agua con la ecuación de Peng Robinson.}\label{fig:gibbs3d}
\end{figure}