\subsection{Energía libre de Gibbs}
	
	El cálculo de la energía libre de gibbs se realiza a partir de la entropía y la entalpía segun la ecuación \ref{eq:gibbs} y su cálculo se demuestra en el código \ref{lst:gibbs}.

	\begin{lstlisting}[label={lst:gibbs},caption={Cálculo de la energía libre de gibbs con la librería \Materia}]
		double gibbs = homogeneous.calculateGibbs();
	\end{lstlisting}

	Las figuras \ref{fig:2dgibbs} y \ref{fig:gibbs3d} muestran diagramas de entalpía creados con ayuda de la librería \Materia.

\begin{figure}[!h]
	\centering	
	\begin{tikzpicture}
	\begin{axis}
	\addplot[blue]table{plotdata/gibbs/lv.dat};
	\end{axis}
	\end{tikzpicture}
	\caption{Diagrama de la energía libre de Gibbs para el agua.}\label{fig:2dgibbs}
\end{figure}

\begin{figure}
	\begin{tikzpicture}
	\begin{axis}[view/h=-165]
	\addplot3[surf,point meta=explicit]table[meta=temperature]{plotdata/gibbs/lv3d.dat};
	\addplot3[surf,point meta=explicit]table[meta=temperature]{plotdata/gibbs/l3d.dat};
	\addplot3[surf,point meta=explicit]table[meta=temperature]{plotdata/gibbs/v3d.dat};
	\end{axis}
	\end{tikzpicture}
	\begin{tikzpicture}
	\begin{axis}[view/h=-225]
	\addplot3[surf]table{plotdata/gibbs/lv3d.dat};
	\addplot3[surf]table{plotdata/gibbs/l3d.dat};
	\addplot3[surf]table{plotdata/gibbs/v3d.dat};
	\end{axis}
	\end{tikzpicture}
	\begin{tikzpicture}
	\begin{axis}[view/h=-120]
	\addplot3[surf]table{plotdata/gibbs/lv3d.dat};
	\addplot3[surf]table{plotdata/gibbs/l3d.dat};
	\addplot3[surf]table{plotdata/gibbs/v3d.dat};
	\end{axis}
	\end{tikzpicture}
	\caption{Diagramas de la energía libre de Gibbs para el agua.}\label{fig:gibbs3d}
\end{figure}