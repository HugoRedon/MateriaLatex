\subsection{Presión de Burbuja}\label{subsec:bubblepressure}

	El cálculo de presión de burbuja para una mezcla se realiza como se indica en la figura \ref{fig:bubblepressure}. Usando un objeto del tipo `HeterogeneousMixture' de la librería \Materia el cálculo se realiza como se muestra en los fragmentos de código \ref{lst:bubblepressure} y \ref{lst:bubblepressureWithEstimate}.

	Si al método no se le proporciona un estimado inicial, como se muestra en el código \ref{lst:bubblepressure} entonces la clase realiza la estimación de la presión de vapor con la ecuación del factor acéntrico \ref{eq:pressureacentricfactor}. 



	\begin{lstlisting}[label={lst:bubblepressureWithEstimate},caption={Cálculo de la presión de burbuja proporcionando un estimado inicial.}]

		heterogeneousMixture.setZFraction(compound1,molarFraction1);
		//...... asignar la fracción molar para todos los compuestos

		heterogeneousMixture.setTemperature(temperature);
		heterogeneousMixture.bubblePressure(
					pressureEstimate,vaporEstimatedFractions);
		double pressure = heterogeneousMixture.getPressure();
	\end{lstlisting}


	\begin{lstlisting}[label={lst:bubblepressure},caption={Cálculo de la presión de burbuja.}]
		heterogeneousMixture.setZFraction(compound1,molarFraction1);
		//...... asignar la fracción molar para todos los compuestos

		heterogeneousMixture.setTemperature(temperature);
		heterogeneousMixture.bubblePressure();
		double pressure = heterogeneousMixture.getPressure();
	\end{lstlisting}

	La figura \ref{fig:bubblepressure3d} muestra los planos obtenidos del líquido saturado y del vapor saturado por medio del cálculo de presión de burbuja para el sistema metanol-agua con la ecuación de PRSV.

\begin{figure}[p]
\centering
	\begin{tikzpicture}
	\begin{axis}[view/v=-15,colorbar left,xlabel={Fracción Molar del metanol},ylabel=\temperature,zlabel=\pressure,colorbar style={ylabel=\temperature,
        yticklabel style={
            text width=2.5em,
            align=right}}]
		\addplot3[surf,point meta=explicit,shader=interp]table[meta=temperature, y=temperature , x=liquidFraction, z=pressure]{plotdata/mixhet/hp.dat};
		\addplot3[surf,point meta=explicit,shader=interp]table[meta=temperature, y=temperature , x=vaporFraction, z=pressure]{plotdata/mixhet/hp.dat};
	\end{axis}
	\end{tikzpicture}


	\begin{tikzpicture}
	\begin{axis}[view/v=-220,xlabel={Fracción Molar del metanol},ylabel=\temperature,zlabel=\pressure]
	\addplot3[surf,point meta=explicit,shader=interp]table[meta=temperature, y=temperature , x=liquidFraction, z=pressure]{plotdata/mixhet/hp.dat};
	\addplot3[surf,point meta=explicit,shader=interp]table[meta=temperature, y=temperature , x=vaporFraction, z=pressure]{plotdata/mixhet/hp.dat};
	\end{axis}
	\end{tikzpicture}


	\begin{tikzpicture}
	\begin{axis}[view/v=-115,xlabel={Fracción Molar del metanol},ylabel=\temperature,zlabel=\pressure]
	\addplot3[surf,point meta=explicit,shader=interp]table[meta=temperature, y=temperature , x=liquidFraction, z=pressure]{plotdata/mixhet/hp.dat};
	\addplot3[surf,point meta=explicit,shader=interp]table[meta=temperature, y=temperature , x=vaporFraction, z=pressure]{plotdata/mixhet/hp.dat};
	\end{axis}
	\end{tikzpicture}
	\caption{Diagramas tridimensionales de presión-composición-temperatura para el sistema metanol-agua. En la figura podemos observar los planos formados por el líquido y el vapor saturado obtenidos con el cálculo de presión de burbuja. Los diagramas son la misma figura vista desde diferentes angulos.}\label{fig:bubblepressure3d}
\end{figure}
