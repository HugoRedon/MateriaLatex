\subsection{Presión de Burbuja}\label{subsec:bubblepressure}

	El cálculo de presión de burbuja para una mezcla se realiza como se indica en la figura \ref{fig:bubblepressure}. Usando un objeto del tipo `HeterogeneousMixture' de la librería \Materia el cálculo se realiza como se muestra en los fragmentos de código \ref{lst:bubblepressure} y \ref{lst:bubblepressureWithEstimate}.

	Si al método no se le proporciona un estimado inicial, como se muestra en el código \ref{lst:bubblepressure} entonces la clase realiza la estimación de la presión de vapor con la ecuación del factor acéntrico \ref{eq:pressureacentricfactor}. 



	\begin{lstlisting}[label={lst:bubblepressureWithEstimate},caption={Cálculo de la presión de burbuja proporcionando un estimado inicial.}]
		heterogeneousMixture.setTemperature(temperature);
		heterogeneousMixture.bubblePressure(pressureEstimate);
		double pressure = heterogeneousMixture.getPressure();
	\end{lstlisting}


	\begin{lstlisting}[label={lst:bubblepressure},caption={Cálculo de la presión de burbuja.}]
		heterogeneousMixture.setTemperature(temperature);
		heterogeneousMixture.bubblePressure();
		double pressure = heterogeneousMixture.getPressure();
	\end{lstlisting}

\begin{figure}\centering
	\begin{tikzpicture}
	\begin{axis}[view/v=-15,colorbar left,xlabel={Fracción Molar},ylabel=\temperature,zlabel=\pressure,colorbar style={ylabel=\temperature,
        yticklabel style={
            text width=2.5em,
            align=right}}]
		\addplot3[surf,point meta=explicit,shader=interp]table[meta=temperature, y=temperature , x=liquidFraction, z=pressure]{plotdata/mixhet/hp.dat};
		\addplot3[surf,point meta=explicit,shader=interp]table[meta=temperature, y=temperature , x=vaporFraction, z=pressure]{plotdata/mixhet/hp.dat};
	\end{axis}
	\end{tikzpicture}


	\begin{tikzpicture}
	\begin{axis}[view/v=-220,xlabel={Fracción Molar},ylabel=\temperature,zlabel=\pressure]
	\addplot3[surf,point meta=explicit,shader=interp]table[meta=temperature, y=temperature , x=liquidFraction, z=pressure]{plotdata/mixhet/hp.dat};
	\addplot3[surf,point meta=explicit,shader=interp]table[meta=temperature, y=temperature , x=vaporFraction, z=pressure]{plotdata/mixhet/hp.dat};
	\end{axis}
	\end{tikzpicture}


	\begin{tikzpicture}
	\begin{axis}[view/v=-115,xlabel={Fracción Molar},ylabel=\temperature,zlabel=\pressure]
	\addplot3[surf,point meta=explicit,shader=interp]table[meta=temperature, y=temperature , x=liquidFraction, z=pressure]{plotdata/mixhet/hp.dat};
	\addplot3[surf,point meta=explicit,shader=interp]table[meta=temperature, y=temperature , x=vaporFraction, z=pressure]{plotdata/mixhet/hp.dat};
	\end{axis}
	\end{tikzpicture}
	\caption{Diagramas tridimensionales de presión-composición-temperatura para el sistema metanol-agua.}\label{fig:bubblepressure3d}
\end{figure}

\begin{figure}
	\begin{tikzpicture}[nodes={draw, fill=white,align=center},row sep=0.3cm,column sep=0.5cm] ]

	\node(init){Inicio};
	\node[below of=init,below] (lab)  {Lectura de \textbf{Datos} \\$T,x_1, x_2,\ldots, x_{nc}$};
	\node[below of=lab,below=0.2cm](estim){Estimado inicial de\\ las incognitas\\$P,y_1,y_2,\ldots,y_{nc}$};
	\node[below of=estim,below=0.6cm](relations){Cálculo de las\\ razones de equilibrio\\
	$K_i = \frac{ \hat{\phi}_i^L }{\hat{ \phi}_i^V}$};
	\node[below of=relations,below = 0.6cm](error){Cálculo de la función Error\\$ S_y =\sum_{i=1}^{nc} K_i x_i $\\$ E = {S_y}-1 $};

	\node[below of=error,below=0.6cm](criteria){$|E| \leq 1\cdot10^{-4}\quad \text{o} \quad  1\cdot 10^{-5}$};
	\node[below of=criteria,below=0.4cm](pressureIncrement){Incrementar la presión\\$P^* = P + \Delta P$\\
	$\Delta P = 0.001  \quad \text{o} \quad 0.0001 K$};

	\node[below of=pressureIncrement,below=0.4cm](relationsWithIncrement){Cálculo de las Razones\\ de Equilibrio con $P^*$
	\\$K_i^* = \frac{ \hat{\phi}_i^L }{ \hat{\phi}_i^V}$};

	\node[right of=criteria	,right=2.2cm](end){Fin};


	\node[right of=pressureIncrement, right=3cm](errorWithIncrement){Cálculo de la función Error \\ con $P^*$\\
	$ S_y^* =\sum_{i=1}^{nc} K_i^* x_i $\\$ E^* = S_y^* -1 $};

	\node[right of=relations, right = 3cm](newValues){Cálculo de las nuevas estimaciones \\ de las \textbf{Incógnitas}\\
	\begin{minipage}{0.2\linewidth}
	\begin{equation*}
	T_{nueva} = \frac{P P^* \left(E^*-E\right)}{P^*E^*-P E}
	\end{equation*}
	\begin{equation*}
	 S_y =\sum_{i=1}^{nc} K_i x_i \\
	 \end{equation*}
	 \begin{equation*}
	\left(y_i\right)_{nueva} = \frac{K_i x_i}{S_y}
	\end{equation*}
	\end{minipage}
	};


	\draw[-latex] (init)--(lab);
	\draw[-latex] (lab)--(estim);
	\draw[-latex] (estim)--(relations);
	\draw[-latex] (relations)--(error);
	\draw[-latex] (error)--(criteria);
	\draw[-latex] (criteria)--node[fill=none,draw=none,above]{SI}(end);
	\draw[-latex] (criteria)--node[fill=none,draw=none,left]{NO}(pressureIncrement);
	\draw[-latex] (pressureIncrement)--(relationsWithIncrement);
	\draw[-latex] (relationsWithIncrement)--(errorWithIncrement);
	\draw[-latex] (errorWithIncrement)--(newValues);
	\draw[-latex] (newValues)--(relations);

	\end{tikzpicture}
	\caption{Algoritmo para el cálculo de presión de burbuja}\label{fig:bubblepressure}
\end{figure}