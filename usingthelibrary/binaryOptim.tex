\definecolor{amber}{rgb}{1.0, 0.49, 0.0}

\newcommand{\binaryDiagram}[1] {
\begin{tikzpicture}
\begin{axis}[xlabel={Fración molar del metanol},ylabel=\temperature,legend entries={Experimentales líquido,Experimentales vapor,Calculados líquido,Calculados Vapor},legend style={at={(1.03,0.5)},anchor=west,font=\footnotesize}]
\addplot[amber,only marks,mark size = 1pt]table[x=x1, y=expTemp]{#1};
\addplot[blue,only marks,mark size = 1pt]table[x=yExp, y=expTemp]{#1};
\addplot[green,thick] table[x=x1, y=calcTemp]{#1};
\addplot[red,thick] table[x=yCalc, y=calcTemp]{#1};
\end{axis}
\end{tikzpicture}
 }
\newcommand{\alphaDiagram}[1]{
\begin{tikzpicture}
\begin{axis}
\addplot[blue,only marks,mark size = 1pt]table[x=temperature, y=expPressure]{#1};
\addplot[red,thick]table[x=temperature, y=calcPressure]{#1};
\end{axis}
\end{tikzpicture}
}



\subsection{Estimación de parámetros Binarios para las reglas de mezclado}\label{subsec:binaryoptim}

	Para realizar las estimación de los parámetros binarios de las reglas de mezclado, es necesaria una lista de datos experimentales de presión, temperatura y fracciones molares del líquido y del vapor.

	La clase `ErrorFunction' realiza el cálculo de la función error, y la clase `NewtonMethodSolver' realiza el algoritmo para la estimación de los parámetros minimizando el valor de la función error.

	Para crear una lista de datos experimentales para una mezcla binaria, es necesario indicar si la experimentación se llevó a cabo de manera isobárica o isotérmica, el tipo de experimentación se indica con el método `setType' y se puede dar como argumento del método las opciones `ExperimentalDataBinaryType.isobaric' ó  `ExperimentalDataBinaryType.isothermic'. Cuando se indican las fracciones molares del líquido y del vapor solo se indican las de un compuesto, por lo tanto se debe decir a la lista de datos cual es el compuesto de referenecia y cual es el de no referencia, ello se indica con los métodos `setReferenceComponent' y `setNonReferenceComponent'.

	Es muy importante que los compuestos de la mezcla sean iguales a los indicados en la lista, sin embargo en este trabajo no se ha implementado una forma para asegurar que esto suceda, esto se deberá realizarse en un trabajo futuro.

	El código \ref{lst:binarydatalist} muestra la creación de la lista de datos experimentales y la asigna a la mezcla heterogénea.

\begin{lstlisting}[caption={Creación de la lista de datos experimentales y la asignación de esta a la mezcla heterogénea.},label={lst:binarydatalist}]
	ExperimentalDataBinaryList blist = 
						new ExperimentalDataBinaryList("Ejemplo lista");
	blist.setSource("Fuente de los datos");
	blist.setType(experimentalDataBinaryType);

	blist.setReferenceComponent(referenceCompound);
	blist.setNonReferenceComponent(nonReferenceCompound);

	//agregar todos los datos 
	blist.addExperimentalDataToList(temperature, pressure, x, y);
	//......

	heterogeneousMixture.getErrorfunction().setExperimental(blist);
\end{lstlisting}


	La figura \ref{fig:binarybefore} muestra el diagrama temperatura-fración molar antes de la estimación de parámetros, para el sistema metanol-agua, con la ecuación PRSV, y la regla de mezclado de Van Der Waals.

\begin{figure}[!h]
	\centering
	\binaryDiagram{plotdata/binaryOptim/methanolWater/error.dat}
	\caption{Diagrama temperatura-fración molar para el sistema metanol-agua, con la ecuación PRSV y la regla de mezclado de Van Der Waals. Los parámetros de la expresión de alfa para los compuestos es igual a cero, y tambien los parámetros binarios de la regla de mezclado}\label{fig:binarybefore}
\end{figure}

	Es necesario realizar la estimación de los parámetros de la expresión de $\alpha$ para cada compuesto antes de la estimación de los parámetros binarios.

	El resultado de la estimación de los parámetros de la expresión de $\alpha$ para cada compuesto es:
\begin{itemize}
	\item Metanol,$ k: -2.7 \cdot 10^{-5}$
	\item Agua,$ k: 0.009276$
\end{itemize}

	La estimación de los parámetros binarios se hace a través de la función error, como se muestra en el código \ref{lst:mixtureMinimize}.

\begin{lstlisting}[label={lst:mixtureMinimize},caption={Estimación de los parámetros binarios de la regla de mezclado minimizando el valor de la función error.}]
	heterogeneousMixture.getErrorfunction().minimize();
\end{lstlisting}

	En la figura \ref{fig:binaryafter} se muestra el resultado de la estimación de parámetros.


%\alphaDiagram{plotdata/binaryOptim/methanolWater/errorWaterAlpha.dat}
%\alphaDiagram{plotdata/binaryOptim/methanolWater/errorWaterAlphaAfterOptim.dat}
%\alphaDiagram{plotdata/binaryOptim/methanolWater/errorMethanolAlpha.dat}
%alphaDiagram{plotdata/binaryOptim/methanolWater/errorMethanolAlphaAfterOptim.dat}

\begin{figure}[!h]
	\centering
	\binaryDiagram{plotdata/binaryOptim/methanolWater/errorAfterOptim.dat}
	\caption{Diagrama temperatura-fración molar para el sistema agua-metanol, con la ecuación PRSV, parámetro binario $kij=-0.0739981$, parámetros de $\alpha$: Metanol $k= -2.7 \cdot 10^{-5}$,Agua $ k= 0.009276$  }\label{fig:binaryafter}
\end{figure}






%\section{Metanol-Dióxido de carbono}



%\begin{tikzpicture}
%\begin{axis}
%	\addplot[blue,only marks]table[x=x, y=pressure]{plotdata/binaryOptim/co2Met/co2Met25.dat};
%	\addplot[red,only marks]table[x=y, y=pressure]{plotdata/binaryOptim/co2Met/co2Met25.dat};
%	\addplot[brown]table[x=x, y=calcPressure]{plotdata/binaryOptim/co2Met/co2Met25.dat};
%	\addplot[green]table[x=ycalc, y=calcPressure]{plotdata/binaryOptim/co2Met/co2Met25.dat};
%\end{axis}
%\end{tikzpicture}