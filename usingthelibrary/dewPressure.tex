\subsection{Presión de Rocío}\label{subsec:dewpressure}

	El cálculo de presión de rocío para una mezcla se realiza como se indica en la figura \ref{fig:dewpressure}. Usando un objeto del tipo `HeterogeneousMixture' de la librería \Materia el cálculo se realiza como se muestra en los fragmentos de código \ref{lst:dewpressure} y \ref{lst:dewpressureWithEstimate}.

	Si al método no se le proporciona un estimado inicial, como se muestra en el código \ref{lst:dewpressure} entonces la clase realiza la estimación de la presión de vapor con la ecuación del factor acéntrico \ref{eq:pressureacentricfactor}. 

	\begin{lstlisting}[label={lst:dewpressureWithEstimate},caption={Cálculo de la presión de rocío proporcionando un estimado inicial.}]

		heterogeneousMixture.setZFraction(compound1,molarFraction1);
		//...... asignar la fracción molar para todos los compuestos

		heterogeneousMixture.setTemperature(temperature);
		heterogeneousMixture.dewPressure(pressureEstimate,liquidEstimatedFractions);
		double pressure = heterogeneousMixture.getPressure();
	\end{lstlisting}

	\begin{lstlisting}[label={lst:dewpressure},caption={Cálculo de la presión de rocío.}]

		heterogeneousMixture.setZFraction(compound1,molarFraction1);
		//...... asignar la fracción molar para todos los compuestos

		heterogeneousMixture.setTemperature(temperature);
		heterogeneousMixture.dewPressure();
		double pressure = heterogeneousMixture.getPressure();
	\end{lstlisting}
