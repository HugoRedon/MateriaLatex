\subsection{Presión de Rocío}\label{subsec:dewpressure}

	El cálculo de presión de rocío para una mezcla se realiza como se indica en la figura \ref{fig:dewpressure}. Usando un objeto del tipo `HeterogeneousMixture' de la librería \Materia el cálculo se realiza como se muestra en los fragmentos de código \ref{lst:dewpressure} y \ref{lst:dewpressureWithEstimate}.

	Si al método no se le proporciona un estimado inicial, como se muestra en el código \ref{lst:dewpressure} entonces la clase realiza la estimación de la presión de vapor con la ecuación del factor acéntrico \ref{eq:pressureacentricfactor}. 

	\begin{lstlisting}[label={lst:dewpressureWithEstimate},caption={Cálculo de la presión de rocío proporcionando un estimado inicial.}]

		heterogeneousMixture.setZFraction(compound1,molarFraction1);
		//...... asignar la fracción molar para todos los compuestos

		heterogeneousMixture.setTemperature(temperature);
		heterogeneousMixture.dewPressure(pressureEstimate);
		double pressure = heterogeneousMixture.getPressure();
	\end{lstlisting}

	\begin{lstlisting}[label={lst:dewpressure},caption={Cálculo de la presión de rocío.}]

		heterogeneousMixture.setZFraction(compound1,molarFraction1);
		//...... asignar la fracción molar para todos los compuestos

		heterogeneousMixture.setTemperature(temperature);
		heterogeneousMixture.dewPressure();
		double pressure = heterogeneousMixture.getPressure();
	\end{lstlisting}

\begin{figure}[!h]
	\begin{tikzpicture}[nodes={draw, fill=white,align=center},row sep=0.3cm,column sep=0.5cm] ]

	\node(init){Inicio};
	\node[below of=init,below] (lab)  {Lectura de \textbf{Datos} \\$T,y_1, y_2,\ldots, y_{nc}$};
	\node[below of=lab,below=0.4](estim){Estimado inicial de\\ las incognitas\\$P,x_1,x_2,\ldots,x_{nc}$};
	\node[below of=estim,below=0.4](relations){Cálculo de las\\ razones de equilibrio\\
	$K_i = \frac{ \hat{\phi}_i^L }{ \hat{ \phi}_i^V}$};
	\node[below of=relations,below = 0.5cm](error){Cálculo de la función Error\\
	$ S_x =\sum_{i=1}^{nc}\frac{y_i}{ K_i} $\\$ E = S_x-1 $};

	\node[below of=error,below=0.4](criteria){$|E| \leq 1\cdot10^{-4}\quad \text{o} \quad  1\cdot 10^{-5}$};
	\node[below of=criteria,below=0.4](tempIncrement){Incrementer la temperatura\\$P^* = P + \Delta P$\\
	$\Delta P = 0.001  \quad \text{o} \quad 0.0001 K$};

	\node[below of=tempIncrement,below=0.4](relationsWithIncrement){Cálculo de las Razones\\ de Equilibrio con $P^*$\\$K_i^* = \frac{ \hat{\phi}_i^L }{\hat{\phi}_i^V}$};

	\node[right of=criteria	,right=2.2cm](end){Fin};


	\node[right of=tempIncrement, right=3cm](errorWithIncrement){Cálculo de la función Error \\ con $P^*$\\
	$ S_x^* =\sum_{i=1}^{nc} \frac{y_i}{K_i^*}  $\\$ E^* = S_x^*-1 $};

	\node[right of=relations, right = 3cm](newValues){Cálculo de las nuevas estimaciones \\ de las \textbf{Incógnitas}\\
	\begin{minipage}{0.2\linewidth}
	\begin{equation*}
	T_{nueva} =P- \frac{E  \left(P^*-P\right)}{E^*-E}
	\end{equation*}
	\begin{equation*}
	 S_x =\sum_{i=1}^{nc}\frac{y_i}{K_i}   \\
	 \end{equation*}
	 \begin{equation*}
	\left(x_i\right)_{nueva} = \frac{y_i}{K_i S_x}
	\end{equation*}
	\end{minipage}
	};


	\draw[-latex] (init)--(lab);
	\draw[-latex] (lab)--(estim);
	\draw[-latex] (estim)--(relations);
	\draw[-latex] (relations)--(error);
	\draw[-latex] (error)--(criteria);
	\draw[-latex] (criteria)--node[fill=none,draw=none,above]{SI}(end);
	\draw[-latex] (criteria)--node[fill=none,draw=none,left]{NO}(tempIncrement);
	\draw[-latex] (tempIncrement)--(relationsWithIncrement);
	\draw[-latex] (relationsWithIncrement)--(errorWithIncrement);
	\draw[-latex] (errorWithIncrement)--(newValues);
	\draw[-latex] (newValues)--(relations);

	\end{tikzpicture}
	\caption{Algoritmo para el cálculo de la presión de rocío.}\label{fig:dewpressure}
\end{figure}