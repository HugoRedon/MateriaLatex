\subsection{Presión de saturación}\label{subsec:saturationpressure}

	El cálculo de presión de saturación se realiza como se indica en la figura \ref{fig:saturationpressure}. Usando un objeto del tipo `HeterogeneousSubstance' de la librería \Materia el cálculo se realiza como se muestra en los fragmentos de código \ref{lst:saturationpressure} y \ref{lst:saturationpressureWithEstimate}.

	Para realizar el cálculo de presión de saturación primero se necesita indicar la temperatura con el método `setTemperature', después se debe llamar el método `saturationPressure' y finalmente leer el resultado de la presión con el método `getPressure'. El estimado inicial se proporciona como argumento del método `saturationPressure' su uso se muestra en el código \ref{lst:saturationpressureWithEstimate}. Si al método no se le proporciona un estimado inicial, como se muestra en el código \ref{lst:saturationpressure} entonces la clase realiza la estimación de la presión de vapor con la ecuación del factor acéntrico \ref{eq:pressureacentricfactor}. 

	\begin{lstlisting}[label={lst:saturationpressureWithEstimate},caption={Cálculo de la presión de saturación proporcionando un estimado inicial.}]
		heterogeneousSubstance.setTemperature(temperature);
		heterogeneousSubstance.saturationPressure(pressureEstimate);
		double pressure = heterogeneousSubstance.getPressure();
	\end{lstlisting}


	\begin{lstlisting}[label={lst:saturationpressure},caption={Cálculo de la presión de saturación.}]
		heterogeneousSubstance.setTemperature(temperature);
		heterogeneousSubstance.saturationPressure();
		double pressure = heterogeneousSubstance.getPressure();
	\end{lstlisting}

\begin{figure}
	\begin{tikzpicture}[nodes={draw, fill=white,align=center},row sep=0.3cm,column sep=0.5cm] ]

	\node(init){Inicio};
	\node[below of=init,below] (lab)  {Lectura de \textbf{Datos} \\$T$};
	\node[below of=lab,below=0.2cm](estim){Estimado inicial de\\ la incognita\\$P$};
	\node[below of=estim,below=0.6cm](relations){Cálculo de la\\ razon de equilibrio\\
	$K = \frac{ \hat{\phi}^L }{\hat{ \phi}^V}$};
	\node[below of=relations,below = 0.6cm](error){Cálculo de la función Error\\$ E = {K}-1 $};

	\node[below of=error,below=0.6cm](criteria){$|E| \leq 1\cdot10^{-4}\quad \text{o} \quad  1\cdot 10^{-5}$};
	\node[below of=criteria,below=0.4cm](pressureIncrement){Incrementar la presión\\$P^* = P + \Delta P$\\
	$\Delta P = 0.001  \quad \text{o} \quad 0.0001 K$};

	\node[below of=pressureIncrement,below=0.4cm](relationsWithIncrement){Cálculo de la Razon\\ de Equilibrio con $P^*$
	\\$K^* = \frac{ \hat{\phi}^L }{ \hat{\phi}^V}$};

	\node[right of=criteria	,right=2.2cm](end){Fin};


	\node[right of=pressureIncrement, right=3cm](errorWithIncrement){Cálculo de la función Error \\ con $P^*$\\$ E^* = K^* -1 $};

	\node[right of=relations, right = 3cm](newValues){Cálculo de la nueva estimacion \\ de la \textbf{Incógnita}\\
	\begin{minipage}{0.2\linewidth}
	\begin{equation*}
	T_{nueva} = \frac{P P^* \left(E^*-E\right)}{P^*E^*-P E}
	\end{equation*} 
	\end{minipage}
	};


	\draw[-latex] (init)--(lab);
	\draw[-latex] (lab)--(estim);
	\draw[-latex] (estim)--(relations);
	\draw[-latex] (relations)--(error);
	\draw[-latex] (error)--(criteria);
	\draw[-latex] (criteria)--node[fill=none,draw=none,above]{SI}(end);
	\draw[-latex] (criteria)--node[fill=none,draw=none,left]{NO}(pressureIncrement);
	\draw[-latex] (pressureIncrement)--(relationsWithIncrement);
	\draw[-latex] (relationsWithIncrement)--(errorWithIncrement);
	\draw[-latex] (errorWithIncrement)--(newValues);
	\draw[-latex] (newValues)--(relations);

	\end{tikzpicture}
	\caption{Algoritmo para el cálculo de presión de saturación.}\label{fig:saturationpressure}
\end{figure}







