\subsection{Presión de saturación}\label{subsec:saturationpressure}

	El cálculo de presión de saturación se realiza como se indica en la figura \ref{fig:saturationpressure}. Usando un objeto del tipo `HeterogeneousSubstance' de la librería \Materia el cálculo se realiza como se muestra en los fragmentos de código \ref{lst:saturationpressure} y \ref{lst:saturationpressureWithEstimate}.

	Si al método no se le proporciona un estimado inicial, como se muestra en el código \ref{lst:saturationpressure} entonces la clase realiza la estimación de la presión de vapor con la ecuación del factor acéntrico \ref{eq:pressureacentricfactor}. 

	\begin{lstlisting}[label={lst:saturationpressureWithEstimate},caption={Cálculo de la presión de saturación proporcionando un estimado inicial.}]
		heterogeneousSubstance.setTemperature(temperature);
		heterogeneousSubstance.saturationPressure(pressureEstimate);
		double pressure = heterogeneousSubstance.getPressure();
	\end{lstlisting}


	\begin{lstlisting}[label={lst:saturationpressure},caption={Cálculo de la presión de saturación.}]
		heterogeneousSubstance.setTemperature(temperature);
		heterogeneousSubstance.saturationPressure();
		double pressure = heterogeneousSubstance.getPressure();
	\end{lstlisting}

	Las figuras \ref{fig:saturationPressure} y \ref{fig:saturationPressure3d} muestran el resultado del cálculo de la presión de saturación para el agua con la ecuación de estado de Peng Robinson.

\begin{figure}[!h]
	\centering	
	\begin{tikzpicture}
	\begin{axis}[xlabel=\temperature,ylabel=\pressure]
	\addplot[blue,smooth]table{plotdata/heterogeneous/et.dat};
	\end{axis}
	\end{tikzpicture}
	\caption{Diagrama de presión de saturación para el agua con la ecuación de estado de Peng Robinson}\label{fig:saturationPressure}
\end{figure}

\begin{figure}[!h]
	\centering
	\begin{tikzpicture}
	\begin{axis}[view/h=-165,xlabel={\temperature},ylabel={\molarVolume},zlabel=\pressure]
	\addplot3[surf,point meta=explicit]table[meta=temperature, x=temperature , y=liquidVolume, z=pressure]{plotdata/heterogeneous/et.dat};
	\addplot3[surf,point meta=explicit]table[meta=temperature, x=temperature , y=vaporVolume, z=pressure]{plotdata/heterogeneous/et.dat};
	\end{axis}
	\end{tikzpicture}
	\caption{Diagrama tridimensional de presión de saturación - temperatura- volumen molar para el agua con la ecuación de estado de Peng Robinson}
	\label{fig:saturationPressure3d}
\end{figure}