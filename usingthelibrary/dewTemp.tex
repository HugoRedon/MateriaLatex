\subsection{Temperatura de Rocío}\label{subsec:dewtemperature}

	El cálculo de temperatura de rocío para una mezcla se realiza como se indica en la figura \ref{fig:dewtemperature}. Usando un objeto del tipo `HeterogeneousMixture' de la librería \Materia el cálculo se realiza como se muestra en los fragmentos de código \ref{lst:dewtemperature} y \ref{lst:dewtemperatureWithEstimate}.

	Si al método no se le proporciona un estimado inicial, como se muestra en el código \ref{lst:dewtemperature} entonces la clase realiza la estimación de la temperatura como se muestra en la figura \ref{fig:dewtemperatureEstimate}. 

	\begin{lstlisting}[label={lst:dewtemperatureWithEstimate},caption={Cálculo de la temperatura de rocío proporcionando un estimado inicial.}]

		heterogeneousMixture.setZFraction(compound1,molarFraction1);
		//...... asignar la fracción molar para todos los compuestos

		heterogeneousMixture.setPressure(pressure);
		heterogeneousMixture.dewTemperature(temeperatureEstimate);
		double temperature = heterogeneousMixture.getTemperature();
	\end{lstlisting}


	\begin{lstlisting}[label={lst:dewtemperature},caption={Cálculo de la temperatura de rocío.}]

		heterogeneousMixture.setZFraction(compound1,molarFraction1);
		//...... asignar la fracción molar para todos los compuestos

		heterogeneousMixture.setPressure(pressure);
		heterogeneousMixture.dewTemperature();
		double temperature = heterogeneousMixture.getTemperature();
	\end{lstlisting}

