 \subsection{Presión}\label{subsec:pressure}
	Un cálculo de presión, se realiza como se muestra en el código \ref{lst:pressureCalculation}.
	\begin{lstlisting}[label=lst:pressureCalculation,caption=Cálculo de presión para un objeto tipo homogeneous]
	double pressure = homogeneous.calculatePressure(temperature, volume);
	\end{lstlisting}

	Como ya se mencionó antes el objeto `Homogeneous' del código \ref{lst:pressureCalculation} puede ser una substancia o una mezcla.

 	En la figura \ref{fig:cubicPressureDiagrams} se muestra un ejemplo de uso del cálculo para realizár gŕaficas de presión.

	\begin{figure}[!h]
	\begin{tabular}{c c}
		\begin{tikzpicture}
		\begin{axis}[width= 0.45 \linewidth,font=\footnotesize,
		xlabel = {Volumen molar $[\frac{m^3}{kmol}]$},
		ylabel = {Presión $[Pa]$}]
		\addplot[blue]table{plotdata/pressurevolume.dat};
		\end{axis}
		\end{tikzpicture}
		&
		\begin{tikzpicture}
		\begin{axis}[width= 0.45 \linewidth,,font=\footnotesize,
		xlabel={Volumen molar $[\frac{m^3}{kmol}]$},
		zlabel={Presión $[Pa]$},
		ylabel={Temperatura $[K]$}]
		\addplot3[surf,
		colormap={blueblack}{color=(white) color=(blue)},
		domain=0:1]table{plotdata/pressurevolumetemperature.dat};
		\end{axis}
		\end{tikzpicture}
	\end{tabular}
	\caption{Diagramas de presión para el heptano usando la eq. de Van Der Waals (u=w=0)} \label{fig:cubicPressureDiagrams}
	\end{figure}


	
