 \subsection{Presión}\label{subsec:pressure}
	Un cálculo de presión, se realiza como se muestra en el código \ref{lst:pressureCalculation}.
	\begin{lstlisting}[label=lst:pressureCalculation,caption=Cálculo de presión para el heptano con la ecuación de estado cúbica de Van Der Waals]
	import termo.eos.Cubic;
	...
	Cubic cubic = new Cubic();
	//parametros de van der waals para el heptano
	double a = 3107000.0;
	double b = 0.2049;

	double volume = 1.5;
	double temperature = 300;
	double pressure = cubic.calculatePressure(temperature, volume, a, b);
	\end{lstlisting}

	\begin{figure}[!h]
	\begin{tabular}{c c}
		\begin{tikzpicture}
		\begin{axis}[width= 0.45 \linewidth,font=\footnotesize,
		xlabel = {Volumen molar $[\frac{m^3}{kg}]$},
		ylabel = {Presión $[Pa]$}]
		\addplot[blue]table{plotdata/pressurevolume.dat};
		\end{axis}
		\end{tikzpicture}
		&
		\begin{tikzpicture}
		\begin{axis}[width= 0.45 \linewidth,,font=\footnotesize,
		xlabel={Volumen molar $[\frac{m^3}{kg}]$},
		zlabel={Presión $[Pa]$},
		ylabel={Temperatura $[K]$}]
		\addplot3[surf,
		colormap={blueblack}{color=(white) color=(blue)},
		domain=0:1]table{plotdata/pressurevolumetemperature.dat};
		\end{axis}
		\end{tikzpicture}
	\end{tabular}
	\caption{Diagramas de presión usando la eq. de Van Der Waals (u=w=0)} \label{fig:cubicPressureDiagrams}
	\end{figure}


	 En la figura \ref{fig:cubicPressureDiagrams} se muestra un ejemplo de uso del cálculo para realizár gŕaficas de presión.
