\subsection{Flash}\label{subsec:flash}


	El cálculo del flash temperatura-presión se realiza como se indica en la figura \ref{fig:flash}. Usando un objeto del tipo `HeterogeneousMixture' de la librería \Materia el cálculo se realiza como se muestra en el fragmento de código \ref{lst:flash}.

	Si no se proporcionan las fracciones molares del líquido y del vapor, entonces estas son estimadas usando la presión de vapor según el factor acéntrico, y con un valor $vF = 0.5$.

\begin{lstlisting}[label={lst:flash},caption={Cálculo del flash temperatura-presión.}]

	//asignar la fraccion molar para todos los compuestos
	heterogeneousMixture.setZFraction(compound1,molarFraction1);
	//...... 

	double vF = heterogeneousMixture.flash(temperature,pressure);

	//leer las fracciones del vapor y del liquido para todos los compuestos
	double x1 = heterogeneousMixture.getLiquid()
					.getReadOnlyFractions().get(compound1);
	double y1 = heterogeneousMixture.getVapor()
					.getReadOnlyFractions().get(compound1);
	//.....
\end{lstlisting}

\begin{lstlisting}[label={lst:flashWithEstimate},caption={Cálculo del flash temperatura-presión proporcionando estimados iniciales}]

	//asignar la fraccion molar para todos los compuestos
	heterogeneousMixture.setZFraction(compound1,molarFraction1);
	//...... 

	double vF = heterogeneousMixture.flash(temperature,pressure,vaporFractions,liquidFractions,vF);

	//leer las fracciones del vapor y del liquido para todos los compuestos
	double x1 = heterogeneousMixture.getLiquid()
					.getReadOnlyFractions().get(compound1);
	double y1 = heterogeneousMixture.getVapor()
					.getReadOnlyFractions().get(compound1);
	//.....
\end{lstlisting}
