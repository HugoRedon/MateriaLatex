\subsection{Volumen molar}\label{subsec:volume}
	
	El código \ref{lst:volume} muestra como calcular el volumen molar para una substancia o mezcla homogénea.  

\begin{lstlisting}[label={lst:volume},caption={Cálculo del factor de double volume = cubic.}]
	double volume = calculateVolume(temperature, pressure, z);
\end{lstlisting}
	
	En la figura \ref{fig:volume} se muestra la relación que tiene el volumen con el factor de compresibilidad a diferentes temperaturas y presiones.

\begin{figure}[!h]
\begin{tabular}{c c}
	\begin{tikzpicture}
	\begin{axis}[width=0.45\linewidth,font=\footnotesize,view/h=-195,
		ylabel= {Volumen reducido},
		xlabel= {Presión reducida},
		zlabel={Factor de compresibilidad z}]%[colormap/hot]
	\addplot3[surf,point meta=explicit] table[meta=rt,x=p,y=vr,z=z]{plotdata/volume/pz_vr.dat};
	\end{axis}
	\end{tikzpicture}
	&
	\begin{tikzpicture}
	\begin{axis}[width=0.45\linewidth,font=\footnotesize,view/h=90,view/v=-0,
		ylabel= {Volumen reducido},
		zlabel={Factor de compresibilidad z}]%[colormap/hot]
	\addplot3[surf,point meta=explicit] table[meta=rt,x=p,y=vr,z=z]{plotdata/volume/pz_vr.dat};
	\end{axis}
	\end{tikzpicture}
\end{tabular}
\caption{Relación entre el factor de compresibilidad y el volumen}\label{fig:volume}
\end{figure}