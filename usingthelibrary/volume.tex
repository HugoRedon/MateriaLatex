\subsection{volumen molar}\label{subsec:volume}
	Al igual que el factor de compresibilidad para el volumen se necesita resolver la ecuación de estado cúbica. Se obtiene primero el factor de compresibilidad como se indica en la sección \ref{subsec:compresibilityFactor}, y después con la ecuación \ref{eq:z} se calcúla el volumen.

	Utilizando un objeto de la clase `Cubic' en el código \ref{lst:volume} se muestra como calcular el volumen. 

\begin{lstlisting}[label={lst:volume},caption={Cálculo del factor de double volume = cubic.}]
	double volume = calculateVolume(temperature, pressure, z);
\end{lstlisting}
	
	En la figura \ref{fig:volume} se muestra la relación que tiene el volumen con el factor de compresibilidad a diferentes temperaturas y presiones.

\begin{figure}[!h]
\begin{tabular}{c c}
	\begin{tikzpicture}
	\begin{axis}[width=0.45\linewidth,font=\footnotesize,view/h=-195,
		ylabel= {Volumen reducido},
		xlabel= {Presión reducida},
		zlabel={Factor de compresibilidad z}]%[colormap/hot]
	\addplot3[surf,point meta=explicit] table[meta=rt,x=p,y=vr,z=z]{plotdata/volume/pz_vr.dat};
	\end{axis}
	\end{tikzpicture}
	&
	\begin{tikzpicture}
	\begin{axis}[width=0.45\linewidth,font=\footnotesize,view/h=90,view/v=-0,
		ylabel= {Volumen reducido},
		zlabel={Factor de compresibilidad z}]%[colormap/hot]
	\addplot3[surf,point meta=explicit] table[meta=rt,x=p,y=vr,z=z]{plotdata/volume/pz_vr.dat};
	\end{axis}
	\end{tikzpicture}
\end{tabular}
\caption{Relación entre el factor de compresibilidad y el volumen}\label{fig:volume}
\end{figure}