\subsection{Entropía}

	El cálculo de la entropía es muy semejante al de la entalpía, es necesario conocer la entropía del gas ideal y la entropía residual.

\subsubsection{Entropía del gas ideal con la ecuación del calor específico}

	El cálculo de la entropía segun el gas ideal, también necesita una ecuación que represente el calor específico. En la sección \ref{sec:cp} se muestran las ecuaciónes de cp incluidas en la librería y como usarlas.

	En el fragmento de código \ref{lst:idealgasentropy} se muestra el cálculo de la entropía del gas ideal según la ecuación \ref{eq:idealgasentropy}. 

	\begin{lstlisting}[label={lst:idealgasentropy},caption={Cálculo de la entropía absoluta del gas ideal.}]
	double idealGasEntropy = homogeneous.calculateIdealGasEntropy();
	\end{lstlisting}

\subsubsection{Entropía real}

	El cálculo de la entalpía se realiza según la ecuación \ref{eq:entropy} y su uso se muestra en el código \ref{lst:entropy}.

\begin{lstlisting}[label={lst:entropy},caption={Cálculo de la entropía absoluta.}]
	double entropy = homogeneous.calculateEntropy()
\end{lstlisting}

	Las figuras \ref{fig:2dentropy} y \ref{fig:entropy3d} muestran diagramas de entalpía creados con ayuda de la librería \Materia.


\begin{figure}[!h]
	\centering	
	\begin{tikzpicture}
	\begin{axis}[xlabel={\entropy},ylabel=\pressure]
	\addplot[blue]table{plotdata/entropy/lv.dat};
	\end{axis}
	\end{tikzpicture}
	\caption{Diagrama de presión-entropía para el agua.}\label{fig:2dentropy}
\end{figure}

\begin{figure}[!h]
	\begin{tikzpicture}
	\begin{axis}[view/h=-165,xlabel={\entropy},ylabel={\molarVolume},zlabel={\pressure},colorbar]
	\addplot3[surf,point meta=explicit]table[meta=temperature]{plotdata/entropy/lv3d.dat};
	\addplot3[surf,point meta=explicit]table[meta=temperature]{plotdata/entropy/l3d.dat};
	\addplot3[surf,point meta=explicit]table[meta=temperature]{plotdata/entropy/v3d.dat};
	\end{axis}
	\end{tikzpicture}
	\begin{tikzpicture}
	\begin{axis}[view/h=-225,xlabel={\entropy},ylabel={\molarVolume},zlabel={\pressure}]
	\addplot3[surf,point meta=explicit]table[meta=temperature]{plotdata/entropy/lv3d.dat};
	\addplot3[surf,point meta=explicit]table[meta=temperature]{plotdata/entropy/l3d.dat};
	\addplot3[surf,point meta=explicit]table[meta=temperature]{plotdata/entropy/v3d.dat};
	\end{axis}
	\end{tikzpicture}
	\begin{tikzpicture}
	\begin{axis}[view/h=-120,xlabel={\entropy},ylabel={\molarVolume},zlabel={\pressure}]
	\addplot3[surf,point meta=explicit]table[meta=temperature]{plotdata/entropy/lv3d.dat};
	\addplot3[surf,point meta=explicit]table[meta=temperature]{plotdata/entropy/l3d.dat};
	\addplot3[surf,point meta=explicit]table[meta=temperature]{plotdata/entropy/v3d.dat};
	\end{axis}
	\end{tikzpicture}
	\caption{Diagramas tridimensionales presión-entropía-`volumen molar' para el agua.}\label{fig:entropy3d}
\end{figure}