\chapter{Algoritmos para el equilibrio Líquido-Vapor}\label{chap:algorithms}

Los siguientes diagramas muestran los algoritmos empleados en la librería \Materia.


%---------------------------saturationTemperature--------------------------------------------------------------
\begin{figure}[!h]
\begin{tikzpicture}[nodes={draw, fill=white,align=center},row sep=1.4cm,column sep=0.7cm] ]

\node(init){Inicio};
\node[below of=init,below] (lab)  {Lectura de \textbf{Datos} \\$P$};
\node[below of=lab,below = 0.4cm](estim){Estimado inicial de\\ la incognita\\$T$};
\node[below of=estim,below = 0.6](relations){Cálculo de la\\ razon de equilibrio\\
$K = \frac{ \hat{\phi}^L }{\hat{ \phi}^V}$};
\node[below of=relations,below = 0.6cm](error){Cálculo de la función Error\\$ E = \ln{K}$};

\node[below of=error,below = 0.6](criteria){$|E| \leq 1\cdot10^{-4}\quad \text{o} \quad  1\cdot 10^{-5}$};
\node[below of=criteria,below = 0.4](tempIncrement){Incrementar la temperatura\\$T^* = T + \Delta T$\\
$\Delta T = 0.1  \quad \text{o} \quad 1.0 K$};

\node[below of=tempIncrement,below = 0.4](relationsWithIncrement){Cálculo de las Razon\\ de Equilibrio con $T^*$
\\$K^* = \frac{ \hat{\phi}^L }{ \hat{\phi}^V}$};

\node[right of=criteria	,right=2.2cm](end){Fin};


\node[right of=tempIncrement, right=3cm](errorWithIncrement){Cálculo de la función Error \\ con $T^*$\\$ E^* = \ln{K^*}$};

\node[right of=relations, right = 3cm](newValues){Cálculo de la nueva estimación \\ de la \textbf{Incógnita}\\
\begin{minipage}{0.2\linewidth}
\begin{equation*}
T_{nueva} = \frac{TT^* \left(E^*-E\right)}{T^*E^*-TE}
\end{equation*}
\end{minipage}
};


\draw[-latex] (init)--(lab);
\draw[-latex] (lab)--(estim);
\draw[-latex] (estim)--(relations);
\draw[-latex] (relations)--(error);
\draw[-latex] (error)--(criteria);
\draw[-latex] (criteria)--node[fill=none,draw=none,above]{SI}(end);
\draw[-latex] (criteria)--node[fill=none,draw=none,left]{NO}(tempIncrement);
\draw[-latex] (tempIncrement)--(relationsWithIncrement);
\draw[-latex] (relationsWithIncrement)--(errorWithIncrement);
\draw[-latex] (errorWithIncrement)--(newValues);
\draw[-latex] (newValues)--(relations);

\end{tikzpicture}
\caption{Algoritmo de temperatura de saturación.}\label{fig:saturationtemperature}
\end{figure}










\begin{figure}[!h]
\begin{tikzpicture}[nodes={draw, fill=white,align=center},row sep=1.4cm,column sep=0.7cm] ]

\node(init){Inicio};
\node[below of=init,below] (lab)  {Lectura de \textbf{Datos} \\$P$};
\node[below of=lab,below = 0.4cm](estim){Estimado inicial de\\ la incognita\\$T=300 K$};
\node[below of=estim,below = 0.6](relations){Cálculo de la presión de vapor según\\ la ecuación del  factor acéntrico con $T$ \\\\
$ P\degree= P_c 10^{\displaystyle\left[\left(-\frac{7}{3}\right) \left(1+\omega \right)  \left(\left(\frac{T_c}{T}\right) - 1 \right) \right]}$};
\node[below of=relations,below = 1.1cm](error){Cálculo de la función Error\\$ E = \ln{\displaystyle\frac{P\degree}{P}}$};

\node[below of=error,below = 0.6](criteria){$|E| \leq 1\cdot10^{-4}\quad \text{o} \quad  1\cdot 10^{-5}$};
\node[below of=criteria,below = 0.4](tempIncrement){Incrementar la temperatura\\$T^* = T + \Delta T$\\
$\Delta T = 0.1  \quad \text{o} \quad 1.0 K$};

\node[below of=tempIncrement,below = 0.4](relationsWithIncrement){Cálculo de la presión de vapor según\\ la ecuación del factor acéntrico con $T^*$ \\\\
${P\degree}^*= P_c 10^{\displaystyle\left[\left(-\frac{7}{3}\right) \left(1+\omega \right)  \left(\left(\frac{T_c}{T^*}\right) - 1 \right) \right]}$};

\node[right of=criteria	,right=2.2cm](end){Fin};


\node[right of=tempIncrement, right=3cm](errorWithIncrement){Cálculo de la función Error\\$ E^* = \ln{\displaystyle\frac{{P\degree}^*}{P}}$};

\node[right of=relations, right = 3.5cm](newValues){Cálculo de la nueva estimación \\ de la \textbf{Incógnita}\\
\begin{minipage}{0.2\linewidth}
\begin{equation*}
T_{nueva} = \frac{TT^* \left(E^*-E\right)}{T^*E^*-TE}
\end{equation*}
\end{minipage}
};


\draw[-latex] (init)--(lab);
\draw[-latex] (lab)--(estim);
\draw[-latex] (estim)--(relations);
\draw[-latex] (relations)--(error);
\draw[-latex] (error)--(criteria);
\draw[-latex] (criteria)--node[fill=none,draw=none,above]{SI}(end);
\draw[-latex] (criteria)--node[fill=none,draw=none,left]{NO}(tempIncrement);
\draw[-latex] (tempIncrement)--(relationsWithIncrement);
\draw[-latex] (relationsWithIncrement)--(errorWithIncrement);
\draw[-latex] (errorWithIncrement)--(newValues);
\draw[-latex] (newValues)--(relations);

\end{tikzpicture}
\caption{Algoritmo de estimación de temperatura de saturación usando la ecuación del factor acéntrico.}
\label{fig:temperatureEstimate}
\end{figure}


%------------------------saturationPressure------------------------------------------------------------------------------------------------------------------


\begin{figure}
	\begin{tikzpicture}[nodes={draw, fill=white,align=center},row sep=0.3cm,column sep=0.5cm] ]

	\node(init){Inicio};
	\node[below of=init,below] (lab)  {Lectura de \textbf{Datos} \\$T$};
	\node[below of=lab,below=0.2cm](estim){Estimado inicial de\\ la incognita\\$P$};
	\node[below of=estim,below=0.6cm](relations){Cálculo de la\\ razon de equilibrio\\
	$K = \frac{ \hat{\phi}^L }{\hat{ \phi}^V}$};
	\node[below of=relations,below = 0.6cm](error){Cálculo de la función Error\\$ E = {K}-1 $};

	\node[below of=error,below=0.6cm](criteria){$|E| \leq 1\cdot10^{-4}\quad \text{o} \quad  1\cdot 10^{-5}$};
	\node[below of=criteria,below=0.4cm](pressureIncrement){Incrementar la presión\\$P^* = P + \Delta P$\\
	$\Delta P = 0.001  \quad \text{o} \quad 0.0001 K$};

	\node[below of=pressureIncrement,below=0.4cm](relationsWithIncrement){Cálculo de la Razon\\ de Equilibrio con $P^*$
	\\$K^* = \frac{ \hat{\phi}^L }{ \hat{\phi}^V}$};

	\node[right of=criteria	,right=2.2cm](end){Fin};


	\node[right of=pressureIncrement, right=3cm](errorWithIncrement){Cálculo de la función Error \\ con $P^*$\\$ E^* = K^* -1 $};

	\node[right of=relations, right = 3cm](newValues){Cálculo de la nueva estimacion \\ de la \textbf{Incógnita}\\
	\begin{minipage}{0.2\linewidth}
	\begin{equation*}
	T_{nueva} = \frac{P P^* \left(E^*-E\right)}{P^*E^*-P E}
	\end{equation*} 
	\end{minipage}
	};


	\draw[-latex] (init)--(lab);
	\draw[-latex] (lab)--(estim);
	\draw[-latex] (estim)--(relations);
	\draw[-latex] (relations)--(error);
	\draw[-latex] (error)--(criteria);
	\draw[-latex] (criteria)--node[fill=none,draw=none,above]{SI}(end);
	\draw[-latex] (criteria)--node[fill=none,draw=none,left]{NO}(pressureIncrement);
	\draw[-latex] (pressureIncrement)--(relationsWithIncrement);
	\draw[-latex] (relationsWithIncrement)--(errorWithIncrement);
	\draw[-latex] (errorWithIncrement)--(newValues);
	\draw[-latex] (newValues)--(relations);

	\end{tikzpicture}
	\caption{Algoritmo para el cálculo de presión de saturación.}\label{fig:saturationpressure}
\end{figure}

%-----------------------------------------bubbleTemp---------------------------------------------

\begin{figure}[!h]
	\begin{tikzpicture}[nodes={draw, fill=white,align=center},row sep=1.4cm,column sep=0.7cm] ]

	\node(init){Inicio};
	\node[below of=init,below] (lab)  {Lectura de \textbf{Datos} \\$P,x_1, x_2,\ldots, x_{nc}$};
	\node[below of=lab,below = 0.4cm](estim){Estimado inicial de\\ las incognitas\\$T,y_1,y_2,\ldots,y_{nc}$};
	\node[below of=estim,below = 0.6](relations){Cálculo de las\\ razones de equilibrio\\
	$K_i = \frac{ \hat{\phi}_i^L }{\hat{ \phi}_i^V}$};
	\node[below of=relations,below = 0.6cm](error){Cálculo de la función Error\\$ S_y =\sum_{i=1}^{nc} K_i x_i $\\$ E = \ln{S_y} $};

	\node[below of=error,below = 0.6](criteria){$|E| \leq 1\cdot10^{-4}\quad \text{o} \quad  1\cdot 10^{-5}$};
	\node[below of=criteria,below = 0.4](tempIncrement){Incrementer la temperatura\\$T^* = T + \Delta T$\\
	$\Delta T = 0.1  \quad \text{o} \quad 1.0 K$};

	\node[below of=tempIncrement,below = 0.4](relationsWithIncrement){Cálculo de las Razones\\ de Equilibrio con $T^*$
	\\$K_i^* = \frac{ \hat{\phi}_i^L }{ \hat{\phi}_i^V}$};

	\node[right of=criteria	,right=2.2cm](end){Fin};


	\node[right of=tempIncrement, right=3cm](errorWithIncrement){Cálculo de la función Error \\ con $T^*$\\
	$ S_y^* =\sum_{i=1}^{nc} K_i^* x_i $\\$ E^* = \ln{S_y^*} $};

	\node[right of=relations, right = 3cm](newValues){Cálculo de las nuevas estimaciones \\ de las \textbf{Incógnitas}\\
	\begin{minipage}{0.2\linewidth}
	\begin{equation*}
	T_{nueva} = \frac{TT^* \left(E^*-E\right)}{T^*E^*-TE}
	\end{equation*}
	\begin{equation*}
	 S_y =\sum_{i=1}^{nc} K_i x_i \\
	 \end{equation*}
	 \begin{equation*}
	\left(y_i\right)_{nueva} = \frac{K_i x_i}{S_y}
	\end{equation*}
	\end{minipage}
	};


	\draw[-latex] (init)--(lab);
	\draw[-latex] (lab)--(estim);
	\draw[-latex] (estim)--(relations);
	\draw[-latex] (relations)--(error);
	\draw[-latex] (error)--(criteria);
	\draw[-latex] (criteria)--node[fill=none,draw=none,above]{SI}(end);
	\draw[-latex] (criteria)--node[fill=none,draw=none,left]{NO}(tempIncrement);
	\draw[-latex] (tempIncrement)--(relationsWithIncrement);
	\draw[-latex] (relationsWithIncrement)--(errorWithIncrement);
	\draw[-latex] (errorWithIncrement)--(newValues);
	\draw[-latex] (newValues)--(relations);

	\end{tikzpicture}
	\caption{Algoritmo para el cálculo de temperatura de burbuja.}\label{fig:bubbletemperature}
\end{figure}










\begin{figure}[!h]
	\begin{tikzpicture}[nodes={draw, fill=white,align=center},row sep=1.4cm,column sep=0.7cm] ]

	\node(init){Inicio};
	\node[below of=init,below] (lab)  {Lectura de \textbf{Datos} \\$P,x_1, x_2,\ldots, x_{nc}$};
	\node[below of=lab,below = 0.4cm](estim){Estimado inicial de\\ la incognita\\$T=300 K$};
	\node[below of=estim,below = 0.6](relations){Cálculo de la presión de vapor con T \\
	$P\degree = \sum_{i=1}^{nc} x_i p\degree_i$\\${p\degree_i}= P_c 10^{\displaystyle\left[\left(-\frac{7}{3}\right) \left(1+\omega \right)  \left(\left(\frac{T_c}{T}\right) - 1 \right) \right]}$};
	\node[below of=relations,below = 0.6cm](error){Cálculo de la función Error\\$ E = \ln{\displaystyle\frac{P\degree}{P}} $};

	\node[below of=error,below = 0.6](criteria){$|E| \leq 1\cdot10^{-4}\quad \text{o} \quad  1\cdot 10^{-5}$};
	\node[below of=criteria,below = 0.4](tempIncrement){Incrementer la temperatura\\$T^* = T + \Delta T$\\
	$\Delta T = 0.1  \quad \text{o} \quad 1.0 K$};

	\node[below of=tempIncrement,below = 0.4](relationsWithIncrement){Cálculo de la presión de vapor con $T^*$ \\
	$P\degree^* = \sum_{i=1}^{nc} x_i p\degree_i^*$\\${p\degree_i}^*= P_c 10^{\displaystyle\left[\left(-\frac{7}{3}\right) \left(1+\omega \right)  \left(\left(\frac{T_c}{T^*}\right) - 1 \right) \right]}$};

	


	\node[right of=relationsWithIncrement, right=5.5cm](errorWithIncrement){Cálculo de la función Error \\ con $T^*$\\$ E^* = \ln{\displaystyle\frac{P\degree^*}{P}} $};

	\node[right of=relations, right = 5.5cm](newValues){Cálculo de las nueva estimacion \\ de la \textbf{Incógnita}\\
	\begin{minipage}{0.2\linewidth}
	\begin{equation*}
	T_{nueva} = \frac{TT^* \left(E^*-E\right)}{T^*E^*-TE}
	\end{equation*}
	\end{minipage}
	};


	\node[right of=criteria,right=2.5cm](newfractions){Cálculo de las\\ fracciones del vapor\\
	\begin{minipage}{0.2\linewidth}

	\begin{equation*}
	 y_i = \frac{p\degree_i x_i}{P} \\
	 \end{equation*}
	\end{minipage}};


	\node[right of=newfractions	,right=1.7cm](end){Fin};


	\draw[-latex] (init)--(lab);
	\draw[-latex] (lab)--(estim);
	\draw[-latex] (estim)--(relations);
	\draw[-latex] (relations)--(error);
	\draw[-latex] (error)--(criteria);
	\draw[-latex] (criteria)--node[fill=none,draw=none,above]{SI}(newfractions);
	\draw[-latex] (newfractions)--(end);
	\draw[-latex] (criteria)--node[fill=none,draw=none,left]{NO}(tempIncrement);
	\draw[-latex] (tempIncrement)--(relationsWithIncrement);
	\draw[-latex] (relationsWithIncrement)--(errorWithIncrement);
	\draw[-latex] (errorWithIncrement)--(newValues);
	\draw[-latex] (newValues)--(relations);

	\end{tikzpicture}
	\caption{Algoritmo para el cálculo de la temperatura de burbuja con la ecuación del factor acéntrico para la presión.}\label{fig:bubbletemperatureEstimate}
\end{figure}

%-------------------------------------dewTemp---------------------------------------------------------------------------------------------------


\begin{figure}[!h]
	\begin{tikzpicture}[nodes={draw, fill=white,align=center},row sep=0.3cm,column sep=0.5cm] ]

	\node(init){Inicio};
	\node[below of=init,below] (lab)  {Lectura de \textbf{Datos} \\$P,y_1, y_2,\ldots, y_{nc}$};
	\node[below of=lab,below](estim){Estimado inicial de\\ las incognitas\\$T,x_1,x_2,\ldots,x_{nc}$};
	\node[below of=estim,below =0.5](relations){Cálculo de las\\ razones de equilibrio\\
	$K_i = \frac{ \hat{\phi}_i^L }{ \hat{ \phi}_i^V}$};
	\node[below of=relations,below = 0.5cm](error){Cálculo de la función Error\\
	$ S_x =\sum_{i=1}^{nc}\frac{y_i}{ K_i} $\\$ E = \ln{S_x} $};

	\node[below of=error,below=0.4](criteria){$|E| \leq 1\cdot10^{-4}\quad \text{o} \quad  1\cdot 10^{-5}$};
	\node[below of=criteria,below=0.4](tempIncrement){Incrementer la temperatura\\$T^* = T + \Delta T$\\
	$\Delta T = 0.1  \quad \text{o} \quad 1.0 K$};

	\node[below of=tempIncrement,below=0.4](relationsWithIncrement){Cálculo de las Razones\\ de Equilibrio con $T^*$\\$K_i^* = \frac{ \hat{\phi}_i^L }{\hat{\phi}_i^V}$};

	\node[right of=criteria	,right=2.2cm](end){Fin};


	\node[right of=tempIncrement, right=3cm](errorWithIncrement){Cálculo de la función Error \\ con $T^*$\\
	$ S_x^* =\sum_{i=1}^{nc} \frac{y_i}{K_i^*}  $\\$ E^* = \ln{S_x^*} $};

	\node[right of=relations, right = 3cm](newValues){Cálculo de las nuevas estimaciones \\ de las \textbf{Incógnitas}\\
	\begin{minipage}{0.2\linewidth}
	\begin{equation*}
	T_{nueva} = \frac{TT^* \left(E^*-E\right)}{T^*E^*-TE}
	\end{equation*}
	\begin{equation*}
	 S_x =\sum_{i=1}^{nc}\frac{y_i}{K_i}   \\
	 \end{equation*}
	 \begin{equation*}
	\left(x_i\right)_{nueva} = \frac{y_i}{K_i S_x}
	\end{equation*}
	\end{minipage}
	};


	\draw[-latex] (init)--(lab);
	\draw[-latex] (lab)--(estim);
	\draw[-latex] (estim)--(relations);
	\draw[-latex] (relations)--(error);
	\draw[-latex] (error)--(criteria);
	\draw[-latex] (criteria)--node[fill=none,draw=none,above]{SI}(end);
	\draw[-latex] (criteria)--node[fill=none,draw=none,left]{NO}(tempIncrement);
	\draw[-latex] (tempIncrement)--(relationsWithIncrement);
	\draw[-latex] (relationsWithIncrement)--(errorWithIncrement);
	\draw[-latex] (errorWithIncrement)--(newValues);
	\draw[-latex] (newValues)--(relations);

	\end{tikzpicture}
	\caption{Algoritmo para el cálculo de temperatura de rocío}\label{fig:dewtemperature}
\end{figure}






\begin{figure}[!h]
	\begin{tikzpicture}[nodes={draw, fill=white,align=center},row sep=1.4cm,column sep=0.7cm] ]

	\node(init){Inicio};
	\node[below of=init,below] (lab)  {Lectura de \textbf{Datos} \\$P,y_1, y_2,\ldots, y_{nc}$};
	\node[below of=lab,below = 0.4cm](estim){Estimado inicial de\\ la incognita\\$T=300 K$};
	\node[below of=estim,below = 0.6](relations){Cálculo de la presión de vapor con T \\
	$P\degree =\frac{1}{\displaystyle \sum_{i=1}^{nc} y_i p\degree_i}$\\
	${p\degree_i}= P_c 10^{\displaystyle\left[\left(-\frac{7}{3}\right) \left(1+\omega \right)  \left(\left(\frac{T_c}{T}\right) - 1 \right) \right]}$};
	\node[below of=relations,below = 1.1cm](error){Cálculo de la función Error\\$ E = \ln{\displaystyle\frac{P\degree}{P}} $};

	\node[below of=error,below = 0.6](criteria){$|E| \leq 1\cdot10^{-4}\quad \text{o} \quad  1\cdot 10^{-5}$};
	\node[below of=criteria,below = 0.4](tempIncrement){Incrementer la temperatura\\$T^* = T + \Delta T$\\
	$\Delta T = 0.1  \quad \text{o} \quad 1.0 K$};

	\node[below of=tempIncrement,below = 0.4](relationsWithIncrement){Cálculo de la presión de vapor con $T^*$ \\
	$P\degree^* = \frac{1}{\displaystyle\sum_{i=1}^{nc} x_i p\degree_i^*}$\\${p\degree_i}^*= P_c 10^{\displaystyle\left[\left(-\frac{7}{3}\right) \left(1+\omega \right)  \left(\left(\frac{T_c}{T^*}\right) - 1 \right) \right]}$};

	


	\node[right of=relationsWithIncrement, right=5.5cm](errorWithIncrement){Cálculo de la función Error \\ con $T^*$\\$ E^* = \ln{\displaystyle\frac{P\degree^*}{P}} $};

	\node[right of=relations, right = 5.5cm](newValues){Cálculo de las nueva estimacion \\ de la \textbf{Incógnita}\\
	\begin{minipage}{0.2\linewidth}
	\begin{equation*}
	T_{nueva} = \frac{TT^* \left(E^*-E\right)}{T^*E^*-TE}
	\end{equation*}
	\end{minipage}
	};


	\node[right of=criteria,right=2.5cm](newfractions){Cálculo de las\\ fracciones del vapor\\
	\begin{minipage}{0.2\linewidth}

	\begin{equation*}
	 x_i = \frac{y_i P}{p\degree_i} 
	 \end{equation*}
	 
	\end{minipage}};


	\node[right of=newfractions	,right=1.7cm](end){Fin};


	\draw[-latex] (init)--(lab);
	\draw[-latex] (lab)--(estim);
	\draw[-latex] (estim)--(relations);
	\draw[-latex] (relations)--(error);
	\draw[-latex] (error)--(criteria);
	\draw[-latex] (criteria)--node[fill=none,draw=none,above]{SI}(newfractions);
	\draw[-latex] (newfractions)--(end);
	\draw[-latex] (criteria)--node[fill=none,draw=none,left]{NO}(tempIncrement);
	\draw[-latex] (tempIncrement)--(relationsWithIncrement);
	\draw[-latex] (relationsWithIncrement)--(errorWithIncrement);
	\draw[-latex] (errorWithIncrement)--(newValues);
	\draw[-latex] (newValues)--(relations);

	\end{tikzpicture}
	\caption{Algoritmo para el cálculo de la temperatura de rocío con la ecuación del factor acéntrico para la presión.}\label{fig:dewtemperatureEstimate}
\end{figure}




%-------------------------------------------bubblepressure-------------------------------------------------------------


\begin{figure}
	\begin{tikzpicture}[nodes={draw, fill=white,align=center},row sep=0.3cm,column sep=0.5cm] ]

	\node(init){Inicio};
	\node[below of=init,below] (lab)  {Lectura de \textbf{Datos} \\$T,x_1, x_2,\ldots, x_{nc}$};
	\node[below of=lab,below=0.2cm](estim){Estimado inicial de\\ las incognitas\\$P,y_1,y_2,\ldots,y_{nc}$};
	\node[below of=estim,below=0.6cm](relations){Cálculo de las\\ razones de equilibrio\\
	$K_i = \frac{ \hat{\phi}_i^L }{\hat{ \phi}_i^V}$};
	\node[below of=relations,below = 0.6cm](error){Cálculo de la función Error\\$ S_y =\sum_{i=1}^{nc} K_i x_i $\\$ E = {S_y}-1 $};

	\node[below of=error,below=0.6cm](criteria){$|E| \leq 1\cdot10^{-4}\quad \text{o} \quad  1\cdot 10^{-5}$};
	\node[below of=criteria,below=0.4cm](pressureIncrement){Incrementar la presión\\$P^* = P + \Delta P$\\
	$\Delta P = 0.001  \quad \text{o} \quad 0.0001 K$};

	\node[below of=pressureIncrement,below=0.4cm](relationsWithIncrement){Cálculo de las Razones\\ de Equilibrio con $P^*$
	\\$K_i^* = \frac{ \hat{\phi}_i^L }{ \hat{\phi}_i^V}$};

	\node[right of=criteria	,right=2.2cm](end){Fin};


	\node[right of=pressureIncrement, right=3cm](errorWithIncrement){Cálculo de la función Error \\ con $P^*$\\
	$ S_y^* =\sum_{i=1}^{nc} K_i^* x_i $\\$ E^* = S_y^* -1 $};

	\node[right of=relations, right = 3cm](newValues){Cálculo de las nuevas estimaciones \\ de las \textbf{Incógnitas}\\
	\begin{minipage}{0.2\linewidth}
	\begin{equation*}
	T_{nueva} = \frac{P P^* \left(E^*-E\right)}{P^*E^*-P E}
	\end{equation*}
	\begin{equation*}
	 S_y =\sum_{i=1}^{nc} K_i x_i \\
	 \end{equation*}
	 \begin{equation*}
	\left(y_i\right)_{nueva} = \frac{K_i x_i}{S_y}
	\end{equation*}
	\end{minipage}
	};


	\draw[-latex] (init)--(lab);
	\draw[-latex] (lab)--(estim);
	\draw[-latex] (estim)--(relations);
	\draw[-latex] (relations)--(error);
	\draw[-latex] (error)--(criteria);
	\draw[-latex] (criteria)--node[fill=none,draw=none,above]{SI}(end);
	\draw[-latex] (criteria)--node[fill=none,draw=none,left]{NO}(pressureIncrement);
	\draw[-latex] (pressureIncrement)--(relationsWithIncrement);
	\draw[-latex] (relationsWithIncrement)--(errorWithIncrement);
	\draw[-latex] (errorWithIncrement)--(newValues);
	\draw[-latex] (newValues)--(relations);

	\end{tikzpicture}
	\caption{Algoritmo para el cálculo de presión de burbuja}\label{fig:bubblepressure}
\end{figure}


%--------------------------------------------dewPressure--------------------------------------------------------------------------------------------------


\begin{figure}[!h]
	\begin{tikzpicture}[nodes={draw, fill=white,align=center},row sep=0.3cm,column sep=0.5cm] ]

	\node(init){Inicio};
	\node[below of=init,below] (lab)  {Lectura de \textbf{Datos} \\$T,y_1, y_2,\ldots, y_{nc}$};
	\node[below of=lab,below=0.4](estim){Estimado inicial de\\ las incognitas\\$P,x_1,x_2,\ldots,x_{nc}$};
	\node[below of=estim,below=0.4](relations){Cálculo de las\\ razones de equilibrio\\
	$K_i = \frac{ \hat{\phi}_i^L }{ \hat{ \phi}_i^V}$};
	\node[below of=relations,below = 0.5cm](error){Cálculo de la función Error\\
	$ S_x =\sum_{i=1}^{nc}\frac{y_i}{ K_i} $\\$ E = S_x-1 $};

	\node[below of=error,below=0.4](criteria){$|E| \leq 1\cdot10^{-4}\quad \text{o} \quad  1\cdot 10^{-5}$};
	\node[below of=criteria,below=0.4](tempIncrement){Incrementer la temperatura\\$P^* = P + \Delta P$\\
	$\Delta P = 0.001  \quad \text{o} \quad 0.0001 K$};

	\node[below of=tempIncrement,below=0.4](relationsWithIncrement){Cálculo de las Razones\\ de Equilibrio con $P^*$\\$K_i^* = \frac{ \hat{\phi}_i^L }{\hat{\phi}_i^V}$};

	\node[right of=criteria	,right=2.2cm](end){Fin};


	\node[right of=tempIncrement, right=3cm](errorWithIncrement){Cálculo de la función Error \\ con $P^*$\\
	$ S_x^* =\sum_{i=1}^{nc} \frac{y_i}{K_i^*}  $\\$ E^* = S_x^*-1 $};

	\node[right of=relations, right = 3cm](newValues){Cálculo de las nuevas estimaciones \\ de las \textbf{Incógnitas}\\
	\begin{minipage}{0.2\linewidth}
	\begin{equation*}
	T_{nueva} =P- \frac{E  \left(P^*-P\right)}{E^*-E}
	\end{equation*}
	\begin{equation*}
	 S_x =\sum_{i=1}^{nc}\frac{y_i}{K_i}   \\
	 \end{equation*}
	 \begin{equation*}
	\left(x_i\right)_{nueva} = \frac{y_i}{K_i S_x}
	\end{equation*}
	\end{minipage}
	};


	\draw[-latex] (init)--(lab);
	\draw[-latex] (lab)--(estim);
	\draw[-latex] (estim)--(relations);
	\draw[-latex] (relations)--(error);
	\draw[-latex] (error)--(criteria);
	\draw[-latex] (criteria)--node[fill=none,draw=none,above]{SI}(end);
	\draw[-latex] (criteria)--node[fill=none,draw=none,left]{NO}(tempIncrement);
	\draw[-latex] (tempIncrement)--(relationsWithIncrement);
	\draw[-latex] (relationsWithIncrement)--(errorWithIncrement);
	\draw[-latex] (errorWithIncrement)--(newValues);
	\draw[-latex] (newValues)--(relations);

	\end{tikzpicture}
	\caption{Algoritmo para el cálculo de la presión de rocío.}\label{fig:dewpressure}
\end{figure}
%--------------------------flash-------------------------------------------------------------------------------------------------

\begin{figure}[!h]
	\begin{tikzpicture}[nodes={draw, fill=white,align=center},row sep=0.3cm,column sep=0.5cm] ]

	\node(init){Inicio};
	\node[below of=init,below] (lab)  {Lectura de \textbf{Datos} \\$T,P,z_1, z_2,\ldots, z_{nc}$};
	\node[below of=lab,below=0.4](estim){Estimado inicial de\\ las incognitas\\$V/F,x_1,x_2,\ldots,x_{nc}$\\$y_1,y_2 \ldots, y_{nc}$};
	\node[below of=estim,below=0.4](relations){Cálculo de las\\ razones de equilibrio\\
	$K_i = \frac{ \hat{\phi}_i^L }{ \hat{ \phi}_i^V}$};
	\node[below of=relations,below = 0.4cm](error){Cálculo de la función Error\\
	$ \zeta =\sum_{i=1}^{nc}\left| x_i \hat{\varphi}_i^L - y_i \hat{\varphi}_i^V \right|$};

	\node[below of=error,below=0.4](criteria){$|\zeta| \leq 1\cdot10^{-4}\quad \text{o} \quad  1\cdot 10^{-5}$};


	\node[below of=criteria,below=0.4](rachford){Cálculo de $V/F$ con Rachford-Rice:\\ 
	\begin{minipage}{0.5\linewidth}
	\begin{equation*}
	S = \sum_{i=1}^{nc}\frac{z_i\left(K_i - 1\right)}{1+ \frac{V}{F}\left(K_i-1\right)}
	\end{equation*}
	Encontrar $V/F$ tal que $S = 0$\\
	Método de Newton-Raphson\\
	\begin{equation*}
	\acute{S} = \sum_{i=1}^{nc} \frac{-z_i \left(K_i - 1\right)^2}{\left[1+ \frac{V}{F} \left(K_i-1\right)\right]^2}
	\end{equation*}
	\begin{equation*}
	\left(\frac{V}{F}\right)_{nueva} = \left(\frac{V}{F}\right) - \frac{S}{\acute{S}}
	\end{equation*}
	\end{minipage}};

	\node[right of=criteria	,right=2.2cm](end){Fin};


	\node[right of=estim, right = 3cm](newValues){Cálculo de las nuevas estimaciones \\ de las \textbf{Incógnitas}\\
	\begin{minipage}{0.3\linewidth}
	\begin{equation*}
	\acute{x_i} = \frac{z_i}{1 + \frac{V}{F}\left(K_i-1\right)}
	\end{equation*}
	\begin{equation*}
	 S_x =\sum_{i=1}^{nc}\acute{x}_i \qquad x_i = \frac{\acute{x_i}}{S_x}
	\end{equation*}
	\begin{equation*}
	\acute{y_i} = \acute{x_i} K_i = \frac{z_i K_i}{1+\frac{V}{F}\left(K_i - 1\right)}
	\end{equation*}
	\begin{equation*}
	 S_y =\sum_{i=1}^{nc}\acute{y}_i \qquad y_i = \frac{\acute{y_i}}{S_y}
	\end{equation*}
	\end{minipage}
	};


	\draw[-latex] (init)--(lab);
	\draw[-latex] (lab)--(estim);
	\draw[-latex] (estim)--(relations);
	\draw[-latex] (relations)--(error);
	\draw[-latex] (error)--(criteria);
	\draw[-latex] (criteria)--node[fill=none,draw=none,above]{SI}(end);
	\draw[-latex] (criteria)--node[fill=none,draw=none,left]{NO}(rachford);

	\draw[-latex] (rachford)-|(newValues);
	\draw[-latex] (newValues)--(relations);

	\end{tikzpicture}
	\caption{Algoritmo para el cálculo del flash Temperatura-Presión.}\label{fig:flash}
\end{figure}