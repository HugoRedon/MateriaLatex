\chapter{Herramientas}\label{chap:tools}

	Las herramientas que se describen en este capítulo en conjunto con la librería \textbf{Materia} forman la plataforma de desarrollo propuesta en esta tesis. Estas herramientas dan la versatilidad necesaria para el desarrollo de futuras aplicaciónes de modelado y simulación. 

	Las herramientas resuelven las siguientes necesidades:

\begin{itemize}
	\item Java nos permite el desarrollo de los métodos computacionales de una forma sencilla y segura.
	\item JUnit asegura el funcionamiento de la librería sin importar los cambios que sean introducidos al código.
	\item Git permite administrar los cambios entre versiones de una forma segura.
	\item GitHub guarda el código fuente y permite compartir los cambios a través de internet.
	\item Maven permite construir las aplicaciones descargando las proyectos necesarios desde los servidores centrales de Maven.
	\item Netbeans es el ambiente de desarrollo que permite escribir, compilar y ejecutar las aplicaciones escritas.
	\item Licencia `GNU GENERAL PUBLIC LICENSE Version 2' asegura que el código fuente sea libre para todos los usuarios.%usado segun las condiciones deseadas.
	\item Idioma inglés, mantiene una fluidez en la lectura del código fuente.
\end{itemize}

	\section{Java}

		James Gosling \footnote{Considerado creador del lenguaje Java} define a java como: 

		Un lenguaje simple, orientado a objetos, distribuido, interpretado, robusto, de arquitectura neutral, portátil, de alto rendimiento, seguro, multiproceso y dinámico.\cite{java}

		

		\begin{description}
		\item{Simple:} Una de las razones mas importantes por las que se decidió programar la librería en este lenguaje, es que java evita al programador la necesidad de realizar tareas de índole técnico sobre la computadora, como por ejemplo el almacenamiento de los datos en la memoria. Esto permite al programador concentrarse en el area de estudio deseada.

		\item{Orientado a objetos:}
		Sin duda la perspectiva orientada a objetos de java es otro gran atractivo para las ciencias e ingenierías. La división y clasificación de las ramas de estudio permiten concentrarnos en todos y cada uno de los aspectos importantes sobre el tema. De igual manera la division de un programa en objetos nos permite concentrarnos en los aspectos y actividades importantes del programa. Tómese como ejemplo la clasificación de la materia en homogénea y heterogénea, el aspecto importante en esta clasificación es la presencia de uno o más estados de agregación en el sistema, se puede pensar así en un cálculo de equilibrio Líquido-Vapor para un sistema heterogéneo, pero no para un sistema homogéneo.

		\item{Distribuido:}
		Java es una de las principales herramientas para el desarrollo web. La importancia del desarrollo web radica en su capacidad de difusión masiva. La difusión de un servicio o producto es tan importante como su creación. En conjunto con la librería de funciones se ha creado un sitio web donde se exponen algunas de las funciones mas importantes de la librería.

		\item{De arquitectura neutra:}
		Java es multiplataforma, lo cual significa que puede ser ejecutado en la gran mayoría de los sistemas operativos existentes en el mercado actual.

		\end{description}

	\section{JUnit}

		Es un conjunto de clases (framework) que permite realizar la ejecución de clases Java de manera controlada, para poder evaluar si el funcionamiento de cada uno de los métodos de la clase se comporta como se espera.

		 La biblioteca \textbf{Materia} ha sido creada con la idea de expandirse y ser modificad ó extendida por cualquier persona interesada en ello, por lo tanto es muy importante mantener una evidencia del funcionamiento de la librería. El uso de la tecnología JUnit es una forma para asegurar que los cambios introducidos al programa no han afectado el funcionamiento esperado de la libreria.

		Al momento de escribir este trabajo la librería cuenta con mas de 100 pruebas que definen el funcionamiento esperado.

	\section{Git}

		Git es un sistema de gestión y distribución de código fuente. Permite llevar un registro de los cambios realizados, utilizar las diferentes versiones, y compartir los cambios entre usuarios.

	\section{GitHub}

		GitHub es un servicio de depósitos de repositorios Git. Es un sitio donde se puede guardar el código fuente, es muy fácil contribuir al proyecto, compartir los cambios propuestos y es accesible a todo el público.

	\section{Maven}

		Maven es una herramienta de construcción de código, una de sus grandes ventajas es que puede descargar de manera segura versiones compiladas de proyectos con unas pocas lineas de código. En el apéndice \ref{sec:mavenInstall}  se detalla el proceso de uso.

	\section{Netbeans}
		NetBeans es un entorno de desarrollo integrado principalmente diseñado para el lenguaje Java, pero también incluye otros lenguajes como, PHP, C / C++, y HTML5. 

	\section{Openshift}
		OpenShift es una plataforma de programación en la nube orientada a servicios de Red Hat. Una versión para la nube privada se llama OpenShift Enterprise. El software que ejecuta el servicio se encuentra bajo el nombre `OpenShift Origin' de código abierto y está disponible en GitHub.

	\section{Wildfly}
		WildFly, anteriormente conocido como `JavaBeans Open Source Software Application Server' es un servidor de aplicaciones que implementa la plataforma Java, Enterprise Edition. JBoss está escrito en Java y como tal es multiplataforma: utilizable en cualquier sistema operativo que soporte Java

	\section{Licencia `GNU GENERAL PUBLIC LICENSE Version 2'}

		`GNU GENERAL PUBLIC LICENSE' es una licencia libre, sin derechos para software y otro tipo de obras. Pretende garantizar la libertad de compartir y modificar todas las versiones de un programa - para asegurarse de que sigue siendo software libre para todos sus usuarios.
		
	\section{Idioma}

		Existen dos razones por la que se ha elegido el idioma inglés para expresar las funciones de la librería. La primer razon y las mas importante, es que java ha sido escrito en inglés y por lo tanto las estructuras de control y palabras reservadas. Pongamos como ejemplo el siguiente fragmento de código.

\begin{lstlisting}
public boolean isItADog(Pet pet){
	if ( pet.getSpeciesName().equals("Canis lupus familiaris")) {
		return true;
	}else{
		return false;
	}
}
\end{lstlisting}

	El fragmento de código pretende conocer si el nombre de la especie de la mascota es el nombre científico ``Canis lupus familiaris'', si es asi devuelve verdadero, de lo contrario falso. Veamos ahora la versión en español para el mismo fragmento de código.

\begin{lstlisting}
public boolean esUnPerro(Mascota mascota){
	if ( mascota.getNombreDeLaEspecie().equals("Canis lupus familiaris")){
		return true;
	}else{
		return false;
	}
}
\end{lstlisting}

	En inglés, el order de las palabras denota la diferencia entre una pregunta y una afirmación, de modo que el nombre del método en inglés claramete indica la pregunta ``Is it a Dog?'' cuando en español la diferencia entre la pregunta y la afirmación debe ser escrita con un signo de interrogación ``¿Es un perro?'' sin ellos el nombre del metodo puede parecer la afirmación ``¡Es un perro!'', desgraciadamente el signo de afirmación es un operador en java que indica negación, por lo cuál no puede ser empleado para definier el nombre de un método.

	Nótese el nombre del método ``getNombreDeLaEspecie'', el prefijo ``get'' es una convención en java que significa recuperar, se usa para obtener el valor de la variable que continúe al prefijo, sin el prefijo estaríamos afectando la funcionalidad de la librería.

	Puede apreciarse que la lectura de la línea en ingles es fluida y existe la necesidad de realizar traducciones. Aunque parezca trivial en este ejemplo, en porciones mas grandes de código la diferencia es bastante notable.

	Quiero hacer notar que en ningún momento se intenta hacer una comparación sobre los idiomas, sino señalar el beneficio de la fluidez que se obtiene al no mezclarlos.

	

	