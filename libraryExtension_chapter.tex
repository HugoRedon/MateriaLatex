\chapter{Extender o corregir la librería Materia}\label{chap:libraryExtension}
	
 	Explicando la estructura actual de los objetos y sus relaciones, en este capítulo se mostrará como crear nuevas:

 	\begin{itemize}\itemsep0ex
 		\item{Expresiones de $\alpha$}
 		\item{Reglas de Mezclado}
 		\item{Modelos de actividad} 		
 		\item{Ecuaciones de capacidad calorífica} 		
 	\end{itemize}

 
 	\section{Creación de nuevas expresiones de $\alpha$}\label{sec:newAlphaExpressions}

 	La estructura de una expresión de $\alpha$ depende de dos métodos fundamentales: 

 		\begin{itemize}\itemsep0ex
 			\item El cálculo de la expresión para una temperatura y un compuesto puro.
 			\item El cálculo de la derivada de la expresión con respecto a la temperatura, en específico la librería necesita el cálculo de la siguiente expresión.

 			\begin{equation}
 				\frac{T \partial \alpha}{\alpha \partial T}
 			\end{equation}
 		\end{itemize}

 	Para asegurar el funcionamiento de la estimación de los parámetros de la ecuación se necesitan adicionalmente los métodos:
 		\begin{itemize}\itemsep0ex
 			\item `getParameter'
 			\item `setParámeter'
 			\item `numberOfParameters'
 		\end{itemize}

	.

	Todos los métodos mencionados anteriormente se declaran en la clase abstracta `Alpha' y necesitan ser definidos en la nueva expresión. 

	La nueva clase deberá heredar a la clase `Alpha', y definir los métodos abstractos en ella. 

	Los ambientes de desarrollo como Netbeans o Eclipse pueden generar lo necesario para definir una nueva expresión de $\alpha$. El código \ref{lst:newAlphaCode} muestra la nueva clase `NewAlphaExpresion' que hereda a la clase `Alpha' y define los métodos abstractos.


	\begin{lstlisting}[caption={Esqueleto para la creación de una nueva expresión de $\alpha$ para la librería materia, generado con el ambiente de desarrollo Eclipse},label={lst:newAlphaCode}]
import termo.component.Compound;
import termo.eos.alpha.Alpha;

public class NewAlphaExpresion extends Alpha {

	@Override
	public double TempOverAlphaTimesDerivativeAlphaRespectTemperature(
			double arg0, Compound arg1) {
		// TODO Auto-generated method stub
		return 0;
	}

	@Override
	public double alpha(double arg0, Compound arg1) {
		// TODO Auto-generated method stub
		return 0;
	}

	@Override
	public double getParameter(Compound arg0, int arg1) {
		// TODO Auto-generated method stub
		return 0;
	}

	@Override
	public String getParameterName(int arg0) {
		// TODO Auto-generated method stub
		return null;
	}

	@Override
	public int numberOfParameters() {
		// TODO Auto-generated method stub
		return 0;
	}

	@Override
	public void setParameter(double arg0, Compound arg1, int arg2) {
		// TODO Auto-generated method stub

	}

}

	\end{lstlisting}

	Existen expresiones de $\alpha$ que se definen para temperaturas debajo de la temperatura crítica y la ecuación cambia cuando la temperatura es mayor a la crítica, para tales casos existe la clase `TwoEquationsAlpha' que define los mismos métodos que la clase abstracta `Alpha' pero condiciona la salida de los métodos según las condiciones de temperatura. 

	El código \ref{lst:twoAlphaExample} utiliza dos expresiones de $\alpha$ y las pone dentro de un objeto `TwoEquationsAlpha' esta ecuación es la expresión de Mathias.

	\begin{lstlisting}[caption={Creación de una expresión de $\alpha$ con cambio de ecuación según la temperatura del sistema},label={lst:twoAlphaExample}]
		TwoEquationsAlphaExpression mathias = new TwoEquationsAlphaExpression();    
        mathias.setName(AlphaNames.Mathias);
        CommonAlphaEquation mathiasBelow = new MathiasAlpha();
        
        MathiasAboveTcAlphaEquation mathiasAbove = new MathiasAboveTcAlphaEquation(mathiasBelow);
        
        mathias.setAlphaBelowTc(mathiasBelow);
        mathias.setAlphaAboveTc(mathiasAbove);
        return mathias;     
	\end{lstlisting}

\section{Creación de nuevas reglas de mezclado}\label{sec:newMixingRules}
	
	Los métodos fundamentales de las reglas de mezclado son :

	\begin{itemize}
		\item El cálculo del parámetro a de la ecuación cúbica.
		\item El cálculo del parámetro b de la ecuación cúbica.
		\item La derivada del parámetro a con respecto al número de moles de un compuesto i, en específico se necesita el cálculo de la expresión: 
		\begin{equation}
			\frac{1\partial a N^2}{N \partial N_i}
		\end{equation}
		\item La derivada del parámetro a con respecto a la temperatura, en específico se necesita el cálculo de la expresión: 
		\begin{equation}
			T \frac{\partial a}{\partial T}
		\end{equation}
	\end{itemize}

	Además para el correcto funcionamiento de la estimación de parámetros se necesitan los métodos: 
	\begin{itemize}
		\item `getParameter'
		\item `setParameter'
		\item `numberOfParámeters'
	\end{itemize}

	La diferencia de los métodos `get' y `set' para los parámetros de las expresiones de $\alpha$ y las reglas de mezclado son los argumentos que reciben, ya que para una regla de mezclado son necesarios dos compuestos como argumentos y para la expresión de $\alpha$ no.

	El código \ref{lst:newMixingRule} muestra la clase generada por Eclipse con los métodos necesarios para crear una nueva regla de mezclado.

	\begin{lstlisting}[caption={Esqueleto para la creación de una nueva regla de mezclado},label={lst:newMixingRule}]

import termo.binaryParameter.InteractionParameter;
import termo.component.Compound;
import termo.eos.mixingRule.MixingRule;
import termo.matter.Mixture;
import termo.matter.Substance;

public class NewMixingRule extends MixingRule {

	@Override
	public double a(Mixture arg0) {
		// TODO Auto-generated method stub
		return 0;
	}

	@Override
	public double b(Mixture arg0) {
		// TODO Auto-generated method stub
		return 0;
	}

	@Override
	public double getParameter(Compound arg0, Compound arg1,
			InteractionParameter arg2, int arg3) {
		// TODO Auto-generated method stub
		return 0;
	}

	@Override
	public int numberOfParameters() {
		// TODO Auto-generated method stub
		return 0;
	}

	@Override
	public double oneOverNParcial_aN2RespectN(Substance arg0, Mixture arg1) {
		// TODO Auto-generated method stub
		return 0;
	}

	@Override
	public void setParameter(double arg0, Compound arg1, Compound arg2,
			InteractionParameter arg3, int arg4) {
		// TODO Auto-generated method stub

	}

	@Override
	public double temperatureParcial_a(Mixture arg0) {
		// TODO Auto-generated method stub
		return 0;
	}

}

	\end{lstlisting}

\section{Creación de nuevos modelos de Actividad}\label{sec:newActivityModels}

	Para la reglas de mezclado basadas en la energía libre de Gibbs son necesarios los modelos de actividad, y en la librería Materia es posible crear nuevos modelos.

	La clase abstracta `ActivityModel' declara los métodos necesarios para los modelos de actividad.

	\begin{itemize}
		\item El cálculo del coeficiente de actividad.
		\item El cálculo de la energía libre de Gibbs.
		\item El cálculo de la derivada de la energía libre de Gibbs con respecto a la temperatura.
	\end{itemize}

	Los modelos de actividad también tienen parámetros que se pueden estimar, pero para los modelos actualmente creados se ha utilizado la expresión de $\tau$ :

	\begin{equation}
		\tau = \frac{a_{ij} + b_{ij} T}{R T}
	\end{equation}

	y se han creado los métodos `get' y `set' parámeter para estos parámetros. Por lo cual si el modelo de actividad contiene mas parámetros que pueden ser estimados se deberán sobreescribir los métodos `get' y `set' para los parámetros.

	El código \ref{lst:newActivityModel} muestra la clase generada por Eclipse para la creación de un nuevo modelo de actividad.

	\begin{lstlisting}[caption={Esqueleto de clase para la creación de un nuevo modelo de actividad},label={lst:newActivityModel}]

import java.util.ArrayList;
import java.util.HashMap;

import termo.activityModel.ActivityModel;
import termo.binaryParameter.ActivityModelBinaryParameter;
import termo.component.Compound;
import termo.matter.Mixture;
import termo.matter.Substance;

public class NewActivityModel extends ActivityModel {

	@Override
	public double activityCoefficient(Substance arg0, Mixture arg1) {
		// TODO Auto-generated method stub
		return 0;
	}

	@Override
	public double excessGibbsEnergy(Mixture arg0) {
		// TODO Auto-generated method stub
		return 0;
	}

	@Override
	public double parcialExcessGibbsRespectTemperature(
			ArrayList<Compound> arg0, HashMap<Compound, Double> arg1,
			ActivityModelBinaryParameter arg2, double arg3) {
		// TODO Auto-generated method stub
		return 0;
	}

}
	\end{lstlisting}


\section{Creación de nuevas ecuaciones de capacidad calorífica}\label{sec:newCpEquation}

	A diferencia de los objetos anteriores la capacidad calorífica se define como una interfaz `CpEquation', y para crear una nueva ecuación no es necesario heredar de ella, se debe implementar.


	El código \ref{lst:newCpEquation}

	\begin{lstlisting}[caption={Esqueleto para crear una nueva ecuación de capacidad calorífica}, label={lst:newCpEquation}]

import termo.cp.CpEquation;

public class NewCpEquation implements CpEquation {

	public double Enthalpy(double arg0, double arg1, double arg2) {
		// TODO Auto-generated method stub
		return 0;
	}

	public double cp(double arg0) {
		// TODO Auto-generated method stub
		return 0;
	}

	public String getMathEquation() {
		// TODO Auto-generated method stub
		return null;
	}

	public double idealGasEntropy(double arg0, double arg1, double arg2,
			double arg3, double arg4) {
		// TODO Auto-generated method stub
		return 0;
	}

}
	\end{lstlisting}

