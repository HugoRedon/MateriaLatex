

\chapter{Instalación}


  Durante la primer parte de esta tesis nos enfocaremos en la utilización de la librería para realizar cálculos de interés en la ingeniería quimica, escribiendo pequeños fragmentos de código en java. Para la segunda parte nos enfocaremos en mostrar como modificar y expandir la librería.

	Existen dos formas de utilizar la librería Materia en una aplicación java:
  Descargar el archivo .jar y agregarlo al folder /lib de la aplicación ó desde maven utilizando el archivo pom.xml.

  La librería existe como un archivo .jar y se puede descargar desde la página creada para su difusión (ingenieria-eqpro.rhcloud.com), o desde maven automáticamente desde los servidores de sonatype.

  En el apéndice \ref{sec:manualInstall} se ejemplifica la instalación manual para 




 

